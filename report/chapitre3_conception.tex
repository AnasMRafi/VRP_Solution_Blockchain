%=====================================================================
%            CHAPITRE 3 : CONCEPTION DU SYSTÈME
%=====================================================================

\chapter{Conception du Système}

%---------------------------------------------------------------------
\section{Introduction}
%---------------------------------------------------------------------

Ce chapitre présente la conception détaillée du système RouteChain à travers différents niveaux d'abstraction. Nous commencerons par une vue d'ensemble de l'architecture globale, puis nous utiliserons le langage de modélisation UML pour représenter les différents aspects du système : cas d'utilisation, structure statique et comportement dynamique. Nous terminerons par la conception du schéma de base de données et l'architecture du smart contract.

%---------------------------------------------------------------------
\section{Architecture Globale du Système}
%---------------------------------------------------------------------

\subsection{Architecture Trois-Tiers}

RouteChain adopte une architecture trois-tiers classique, séparant clairement les responsabilités entre la présentation, la logique métier et la persistance des données.

\begin{figure}[H]
\centering
\includegraphics[width=0.85\textwidth]{diagrams/Layers_architecture.pdf}
\caption{Architecture trois-tiers de RouteChain}
\label{fig:architecture_3tiers}
\end{figure}

% COMMENTED OUT: Original TikZ diagram replaced by PNG image
% \begin{tikzpicture}[
%     tier/.style={rectangle, draw=primaryblue, fill=primaryblue!15, thick, minimum width=13cm, minimum height=2.2cm, text centered, rounded corners=8pt, drop shadow},
%     component/.style={rectangle, draw=secondarypurple, fill=white, thick, minimum width=3.2cm, minimum height=1.1cm, text centered, rounded corners=5pt, font=\small\bfseries, drop shadow={opacity=0.3}},
%     arrow/.style={-{Stealth[length=4mm, width=3mm]}, line width=1.5pt, primaryblue}
% ]
%     % Tiers with gradient effect
%     \node[tier] (presentation) at (0,7) {};
%     \node[font=\large\bfseries, primaryblue] at (0,8.2) {Couche Présentation};
%     
%     \node[tier, fill=successgreen!15, draw=successgreen] (logique) at (0,3.5) {};
%     \node[font=\large\bfseries, successgreen] at (0,4.7) {Couche Logique Métier};
%     
%     \node[tier, fill=warningorange!15, draw=warningorange] (donnees) at (0,0) {};
%     \node[font=\large\bfseries, warningorange] at (0,1.2) {Couche Données};
%     
%     % Components - Presentation
%     \node[component] at (-4,6.5) {React 18};
%     \node[component] at (0,6.5) {Leaflet Maps};
%     \node[component] at (4,6.5) {Tailwind CSS};
%     
%     % Components - Logic
%     \node[component, draw=successgreen] at (-4,3) {FastAPI};
%     \node[component, draw=successgreen] at (0,3) {Google OR-Tools};
%     \node[component, draw=successgreen] at (4,3) {Web3.py};
%     
%     % Components - Data
%     \node[component, draw=warningorange] at (-2.5,-0.5) {MongoDB Atlas};
%     \node[component, draw=warningorange] at (2.5,-0.5) {Ethereum};
%     
%     % Arrows
%     \draw[arrow] (0,5.8) -- (0,4.9);
%     \draw[arrow] (0,2.3) -- (0,1.4);
%     
% \end{tikzpicture}

\subsection{Diagramme d'Interaction des Composants}


\begin{figure}[H]
\centering
\includegraphics[width=0.85\textwidth]{diagrams/components_diag.pdf}
\caption{Diagramme d'interaction des composants}
\label{fig:interaction}
\end{figure}

% COMMENTED OUT: Original TikZ diagram replaced by PNG image
% \begin{figure}[H]
% \centering
% \begin{tikzpicture}[
%     box/.style={rectangle, draw=primaryblue, fill=primaryblue!10, thick, minimum width=2.8cm, minimum height=1.4cm, text centered, rounded corners=6pt, font=\small\bfseries, align=center, drop shadow={opacity=0.3}},
%     ext/.style={rectangle, draw=warningorange, fill=warningorange!10, thick, minimum width=2.8cm, minimum height=1.4cm, text centered, rounded corners=6pt, font=\small\bfseries, align=center, drop shadow={opacity=0.3}},
%     arrow/.style={-{Stealth[length=3mm]}, line width=1.2pt, gray!70},
%     label/.style={font=\tiny\bfseries, fill=white, inner sep=2pt}
% ]
%     % Internal components
%     \node[box] (frontend) at (0,0) {Frontend\\React};
%     \node[box] (leaflet) at (0,2.5) {Leaflet\\Maps};
%     \node[box] (backend) at (4.5,0) {Backend\\FastAPI};
%     \node[box, fill=successgreen!10, draw=successgreen] (ortools) at (4.5,2.5) {Google\\OR-Tools};
%     \node[box, fill=successgreen!10, draw=successgreen] (mongodb) at (9,2) {MongoDB\\Atlas};
%     \node[box, fill=secondarypurple!10, draw=secondarypurple] (blockchain) at (9,-2) {Ganache\\Ethereum};
    
%     % External services
%     \node[ext] (ors) at (4.5,-3.5) {OpenRoute\\Service};
%     \node[ext] (nominatim) at (0,-3.5) {Nominatim\\Geocoding};
%     \node[ext] (osm) at (-4,2.5) {OpenStreetMap\\Tiles};
    
%     % Arrows with labels
%     \draw[arrow] (frontend) -- node[label, above] {REST API} (backend);
%     \draw[arrow] (backend) -- node[label, sloped, above] {Motor} (mongodb);
%     \draw[arrow] (backend) -- node[label, sloped, above] {Web3} (blockchain);
%     \draw[arrow] (backend) -- node[label, right] {HTTP} (ors);
%     \draw[arrow] (frontend) -- node[label, left] {HTTP} (nominatim);
%     \draw[arrow] (frontend) -- node[label, right] {uses} (leaflet);
%     \draw[arrow] (leaflet) -- node[label, above] {HTTP} (osm);
%     \draw[arrow] (backend) -- node[label, right] {VRP} (ortools);
    
% \end{tikzpicture}
% \caption{Diagramme d'interaction des composants}
% \label{fig:composants}
% \end{figure}

%---------------------------------------------------------------------
\section{Modélisation UML : Diagrammes de Cas d'Utilisation}
%---------------------------------------------------------------------

\begin{figure}[H]
\centering
\includegraphics[width=1.1\textwidth]{diagrams/route_chain_use_case.pdf}
\caption{Diagramme de cas d'utilisation global}
\label{fig:use_case}
\end{figure}

% % Define stick figure command
% \newcommand{\stickman}[1]{%
%     \begin{scope}[shift={#1}, scale=0.4]
%         \draw[thick] (0,2) circle (0.4); % head
%         \draw[thick] (0,1.6) -- (0,0.5); % body
%         \draw[thick] (-0.5,1.3) -- (0,1.1) -- (0.5,1.3); % arms
%         \draw[thick] (0,0.5) -- (-0.4,0) -- (-0.5,-0.5); % left leg
%         \draw[thick] (0,0.5) -- (0.4,0) -- (0.5,-0.5); % right leg
%     \end{scope}
% }

% \subsection{Diagramme de Cas d'Utilisation Global}

% \begin{figure}[H]
% \centering
% \begin{tikzpicture}[
%     usecase/.style={ellipse, draw=primaryblue, fill=primaryblue!8, thick, minimum width=4cm, minimum height=1.2cm, text centered, font=\small, drop shadow={opacity=0.2}},
%     system/.style={rectangle, draw=primaryblue!50, fill=blue!3, thick, rounded corners=10pt},
%     actorbox/.style={rectangle, draw=none, fill=none},
%     extsystem/.style={rectangle, draw=secondarypurple, fill=secondarypurple!10, thick, minimum width=2cm, minimum height=1.2cm, text centered, font=\small\bfseries, rounded corners=5pt, align=center}
% ]
%     % System boundary - no title inside
%     \draw[system] (1,-5.5) rectangle (10,3.5);
    
%     % Actor - Chauffeur
%     \stickman{(-1, 0.5)}
%     \node[font=\small\bfseries] at (-1, -1) {Chauffeur};
    
%     % Actor - Admin
%     \stickman{(-1, -4)}
%     \node[font=\small\bfseries] at (-1, -5.5) {Admin};
    
%     % External System - Blockchain (right side)
%     \node[extsystem] (blockchain) at (12.5, -3) {Blockchain\\Ethereum};
    
%     % Use cases - positioned to fit within boundary
%     \node[usecase] (auth) at (5.5, 3) {S'authentifier};
%     \node[usecase] (create) at (5.5, 1.5) {Créer une tournée};
%     \node[usecase] (optimize) at (5.5, 0) {Optimiser itinéraire};
%     \node[usecase] (confirm) at (5.5, -1.5) {Confirmer livraison};
%     \node[usecase, fill=secondarypurple!8, draw=secondarypurple] (verify) at (5.5, -3) {Vérifier blockchain};
%     \node[usecase, fill=successgreen!8, draw=successgreen] (stats) at (5.5, -4.5) {Consulter analytics};
    
%     % Connections - Driver (blue)
%     \draw[->, thick, primaryblue!70] (-0.2, 1.4) -- (auth.west);
%     \draw[->, thick, primaryblue!70] (-0.2, 1) -- (create.west);
%     \draw[->, thick, primaryblue!70] (-0.2, 0.6) -- (optimize.west);
%     \draw[->, thick, primaryblue!70] (-0.2, 0.2) -- (confirm.west);
%     \draw[->, thick, secondarypurple!70] (-0.2, -0.2) -- (verify.west);
    
%     % Connections - Admin (orange)
%     \draw[->, thick, warningorange!70] (-0.2, -3.6) -- (auth.west);
%     \draw[->, thick, warningorange!70] (-0.2, -4) -- (stats.west);
%     \draw[->, thick, warningorange!70] (-0.2, -4.4) -- (verify.west);
    
%     % Connection - External system (blockchain)
%     \draw[->, thick, secondarypurple!70] (verify.east) -- (blockchain.west);
    
% \end{tikzpicture}
% \caption{Diagramme de cas d'utilisation global du système RouteChain}
% \label{fig:uc_global}
% \end{figure}

% \subsection{Diagramme de Cas d'Utilisation : Sous-système Chauffeur}

% \begin{figure}[H]
% \centering
% \begin{tikzpicture}[
%     usecase/.style={ellipse, draw=primaryblue, fill=primaryblue!8, thick, minimum width=3.8cm, minimum height=1cm, text centered, font=\footnotesize, drop shadow={opacity=0.2}},
%     system/.style={rectangle, draw=primaryblue!50, fill=blue!3, thick, rounded corners=10pt},
%     include/.style={->, dashed, thick, gray}
% ]
%     % System boundary - no title inside
%     \draw[system] (1,-5) rectangle (9,2.5);
    
%     % Actor
%     \stickman{(-1, -1)}
%     \node[font=\small\bfseries] at (-1, -2.5) {Chauffeur};
    
%     % Use cases - positioned to fit within boundary
%     \node[usecase] (new) at (5, 2) {Créer nouvelle tournée};
%     \node[usecase] (add) at (5, 0.7) {Ajouter points de livraison};
%     \node[usecase] (opt) at (5, -0.6) {Lancer optimisation VRP};
%     \node[usecase] (view) at (5, -1.9) {Visualiser sur carte};
%     \node[usecase] (start) at (5, -3.2) {Démarrer la tournée};
%     \node[usecase, fill=successgreen!8, draw=successgreen] (deliver) at (5, -4.5) {Confirmer livraison};
    
%     % Connections
%     \draw[->, thick, primaryblue!70] (-0.2, -0.6) -- (new.west);
%     \draw[->, thick, primaryblue!70] (-0.2, -0.8) -- (add.west);
%     \draw[->, thick, primaryblue!70] (-0.2, -1) -- (opt.west);
%     \draw[->, thick, primaryblue!70] (-0.2, -1.2) -- (view.west);
%     \draw[->, thick, primaryblue!70] (-0.2, -1.4) -- (start.west);
%     \draw[->, thick, successgreen!70] (-0.2, -1.6) -- (deliver.west);
    
%     % Include relationships
%     \draw[include] (new.south) -- node[font=\tiny, right, pos=0.5] {<<include>>} (add.north);
%     \draw[include] (add.south) -- node[font=\tiny, right, pos=0.5] {<<include>>} (opt.north);
    
% \end{tikzpicture}
% \caption{Diagramme de cas d'utilisation - Sous-système Chauffeur}
% \label{fig:uc_driver}
% \end{figure}

% \subsection{Diagramme de Cas d'Utilisation : Sous-système Administrateur}

% \begin{figure}[H]
% \centering
% \begin{tikzpicture}[
%     usecase/.style={ellipse, draw=secondarypurple, fill=secondarypurple!8, thick, minimum width=3.8cm, minimum height=1cm, text centered, font=\footnotesize, drop shadow={opacity=0.2}},
%     system/.style={rectangle, draw=secondarypurple!50, fill=purple!3, thick, rounded corners=10pt}
% ]
%     % System boundary - no title inside
%     \draw[system] (1,-4.5) rectangle (9,2);
    
%     % Actor
%     \stickman{(-1, -0.5)}
%     \node[font=\small\bfseries] at (-1, -2) {Admin};
    
%     % Use cases - positioned to fit within boundary
%     \node[usecase] (drivers) at (5, 1.5) {Lister tous les chauffeurs};
%     \node[usecase] (role) at (5, 0.2) {Modifier rôle utilisateur};
%     \node[usecase] (routes) at (5, -1.1) {Voir toutes les tournées};
%     \node[usecase, fill=successgreen!8, draw=successgreen] (analytics) at (5, -2.4) {Accéder aux analytics};
%     \node[usecase] (manage) at (5, -3.7) {Gérer clients et dépôts};
    
%     % Connections
%     \draw[->, thick, secondarypurple!70] (-0.2, -0.1) -- (drivers.west);
%     \draw[->, thick, secondarypurple!70] (-0.2, -0.3) -- (role.west);
%     \draw[->, thick, secondarypurple!70] (-0.2, -0.5) -- (routes.west);
%     \draw[->, thick, successgreen!70] (-0.2, -0.7) -- (analytics.west);
%     \draw[->, thick, secondarypurple!70] (-0.2, -0.9) -- (manage.west);
    
% \end{tikzpicture}
% \caption{Diagramme de cas d'utilisation - Sous-système Administrateur}
% \label{fig:uc_admin}
% \end{figure}

%---------------------------------------------------------------------
\section{Modélisation UML : Diagramme de Classes}
%---------------------------------------------------------------------

\begin{figure}[H]
\centering
\includegraphics[width=1\textwidth]{diagrams/routechain_classs.pdf}
\caption{Diagramme des classes}
\label{fig:classes_diag}
\end{figure}

% \subsection{Modèle du Domaine}

% Le diagramme de classes présente l'ensemble des entités, objets de valeur, énumérations et services qui composent le système RouteChain.

% %--- ENTITY CLASSES ---
% \subsubsection{Classes Entités Principales}

% \begin{figure}[H]
% \centering
% \resizebox{\textwidth}{!}{
% \begin{tikzpicture}[
%     classhead/.style={rectangle, draw=primaryblue, fill=primaryblue!20, thick, text width=5cm, text centered, font=\small\bfseries, minimum height=0.7cm},
%     classbody/.style={rectangle, draw=primaryblue, fill=white, thick, text width=5cm, font=\tiny, align=left},
%     classmethods/.style={rectangle, draw=primaryblue, fill=primaryblue!5, thick, text width=5cm, font=\tiny, align=left},
%     stereotype/.style={font=\tiny\itshape, text centered},
%     arrow/.style={->, thick, gray},
%     comp/.style={-{Diamond[fill=black, length=3mm, width=2mm]}, thick},
%     assoc/.style={-, thick},
%     mult/.style={font=\tiny\bfseries, fill=white, inner sep=1pt}
% ]
%     % ========== DRIVER CLASS ==========
%     \node[classhead] (driver_h) at (0, 8) {Driver};
%     \node[stereotype] at (0, 8.5) {<<entity>>};
%     \node[classbody, anchor=north, minimum height=3.5cm] (driver_b) at (0, 7.65) {
%         \texttt{- driver\_id: String}\\
%         \texttt{- full\_name: String}\\
%         \texttt{- email: EmailStr}\\
%         \texttt{- phone: String}\\
%         \texttt{- hashed\_password: String}\\
%         \texttt{- vehicle\_type: VehicleType}\\
%         \texttt{- license\_plate: String}\\
%         \texttt{- max\_capacity: Integer}\\
%         \texttt{- status: DriverStatus}\\
%         \texttt{- role: DriverRole}\\
%         \texttt{- home\_store\_id: String}\\
%         \texttt{- is\_active: Boolean}\\
%         \texttt{- is\_verified: Boolean}\\
%         \texttt{- total\_deliveries: Integer}\\
%         \texttt{- rating: Float}\\
%         \texttt{- created\_at: DateTime}\\
%         \texttt{- updated\_at: DateTime}\\
%         \texttt{- last\_login: DateTime}
%     };
%     \node[classmethods, anchor=north, minimum height=1.2cm] (driver_m) at (0, 4.1) {
%         \texttt{+ register()}\\
%         \texttt{+ login(): Token}\\
%         \texttt{+ updateProfile()}\\
%         \texttt{+ changePassword()}\\
%         \texttt{+ logout()}
%     };
    
%     % ========== ROUTE CLASS ==========
%     \node[classhead, fill=successgreen!20, draw=successgreen] (route_h) at (6.5, 8) {Route};
%     \node[stereotype] at (6.5, 8.5) {<<entity>>};
%     \node[classbody, draw=successgreen, anchor=north, minimum height=3.5cm] (route_b) at (6.5, 7.65) {
%         \texttt{- route\_id: String}\\
%         \texttt{- route\_name: String}\\
%         \texttt{- depot\_location: Location}\\
%         \texttt{- depot\_address: Address}\\
%         \texttt{- driver\_id: String}\\
%         \texttt{- status: RouteStatus}\\
%         \texttt{- delivery\_points: DeliveryPoint[]}\\
%         \texttt{- optimization\_result: OptResult}\\
%         \texttt{- planned\_start\_time: DateTime}\\
%         \texttt{- actual\_start\_time: DateTime}\\
%         \texttt{- completed\_at: DateTime}\\
%         \texttt{- blockchain\_tx\_hash: String}\\
%         \texttt{- blockchain\_block: Integer}\\
%         \texttt{- data\_hash: String}\\
%         \texttt{- created\_at: DateTime}\\
%         \texttt{- updated\_at: DateTime}
%     };
%     \node[classmethods, draw=successgreen, fill=successgreen!5, anchor=north, minimum height=2cm] (route_m) at (6.5, 4.1) {
%         \texttt{+ create()}\\
%         \texttt{+ optimize(): OptResult}\\
%         \texttt{+ start()}\\
%         \texttt{+ complete()}\\
%         \texttt{+ cancel()}\\
%         \texttt{+ delete()}\\
%         \texttt{+ updateStatus(status)}\\
%         \texttt{+ confirmDelivery(point\_id)}\\
%         \texttt{+ exportCSV(): File}\\
%         \texttt{+ exportPDF(): File}\\
%         \texttt{+ verifyBlockchain(): Boolean}
%     };
    
%     % ========== DELIVERYPOINT CLASS ==========
%     \node[classhead, fill=warningorange!20, draw=warningorange] (point_h) at (13, 8) {DeliveryPoint};
%     \node[stereotype] at (13, 8.5) {<<entity>>};
%     \node[classbody, draw=warningorange, anchor=north, minimum height=3.5cm] (point_b) at (13, 7.65) {
%         \texttt{- point\_id: String}\\
%         \texttt{- customer\_name: String}\\
%         \texttt{- address: Address}\\
%         \texttt{- location: Location}\\
%         \texttt{- phone: String}\\
%         \texttt{- instructions: String}\\
%         \texttt{- time\_window\_start: String}\\
%         \texttt{- time\_window\_end: String}\\
%         \texttt{- package\_count: Integer}\\
%         \texttt{- status: DeliveryStatus}\\
%         \texttt{- sequence\_number: Integer}\\
%         \texttt{- estimated\_arrival: DateTime}\\
%         \texttt{- delivered\_at: DateTime}\\
%         \texttt{- driver\_notes: String}\\
%         \texttt{- proof\_of\_delivery: String}
%     };
%     \node[classmethods, draw=warningorange, fill=warningorange!5, anchor=north, minimum height=0.8cm] (point_m) at (13, 4.1) {
%         \texttt{+ confirm(notes, photo)}\\
%         \texttt{+ markFailed(reason)}
%     };
    
%     % ========== CUSTOMER CLASS ==========
%     \node[classhead, fill=secondarypurple!20, draw=secondarypurple] (cust_h) at (0, -0.5) {Customer};
%     \node[stereotype] at (0, 0) {<<entity>>};
%     \node[classbody, draw=secondarypurple, anchor=north, minimum height=2.5cm] (cust_b) at (0, -0.85) {
%         \texttt{- customer\_id: String}\\
%         \texttt{- driver\_id: String}\\
%         \texttt{- name: String}\\
%         \texttt{- email: EmailStr}\\
%         \texttt{- phone: String}\\
%         \texttt{- company: String}\\
%         \texttt{- address: CustomerAddress}\\
%         \texttt{- tags: String[]}\\
%         \texttt{- notes: String}\\
%         \texttt{- delivery\_count: Integer}\\
%         \texttt{- last\_delivery: DateTime}\\
%         \texttt{- created\_at: DateTime}\\
%         \texttt{- updated\_at: DateTime}
%     };
%     \node[classmethods, draw=secondarypurple, fill=secondarypurple!5, anchor=north, minimum height=0.8cm] (cust_m) at (0, -3.4) {
%         \texttt{+ create()}\\
%         \texttt{+ update()}\\
%         \texttt{+ delete()}\\
%         \texttt{+ getDeliveryHistory()}
%     };
    
%     % ========== DEPOT CLASS ==========
%     \node[classhead] (depot_h) at (6.5, -0.5) {Depot};
%     \node[stereotype] at (6.5, 0) {<<entity>>};
%     \node[classbody, anchor=north, minimum height=2.5cm] (depot_b) at (6.5, -0.85) {
%         \texttt{- depot\_id: String}\\
%         \texttt{- driver\_id: String}\\
%         \texttt{- name: String}\\
%         \texttt{- location: DepotLocation}\\
%         \texttt{- is\_default: Boolean}\\
%         \texttt{- capacity: Integer}\\
%         \texttt{- operating\_hours: String}\\
%         \texttt{- contact\_phone: String}\\
%         \texttt{- notes: String}\\
%         \texttt{- routes\_count: Integer}\\
%         \texttt{- created\_at: DateTime}\\
%         \texttt{- updated\_at: DateTime}
%     };
%     \node[classmethods, anchor=north, minimum height=0.8cm] (depot_m) at (6.5, -3.4) {
%         \texttt{+ create()}\\
%         \texttt{+ update()}\\
%         \texttt{+ delete()}\\
%         \texttt{+ setDefault()}
%     };
    
%     % ========== RELATIONSHIPS ==========
%     \draw[assoc] (driver_m.east) -- node[mult, above, pos=0.15] {1} node[mult, above, pos=0.85] {0..*} (route_m.west);
%     \draw[comp] (route_b.east) -- node[mult, above, pos=0.15] {1} node[mult, above, pos=0.85] {5..20} (point_b.west);
%     \draw[assoc] (driver_m.south) -- ++(0,-0.3) -| node[mult, left, pos=0.25] {1} (cust_h.north) node[mult, right, pos=0.95] {0..*};
%     \draw[assoc] (driver_m.south) -- ++(0,-0.3) -| node[mult, left, pos=0.25] {1} (depot_h.north) node[mult, right, pos=0.95] {0..*};
%     \draw[assoc] (route_m.south) -- ++(0,-0.5) -| node[mult, left, pos=0.25] {0..*} (depot_h.north) node[mult, right, pos=0.95] {1};
    
% \end{tikzpicture}
% }
% \caption{Diagramme de classes - Entités principales du domaine}
% \label{fig:class_diagram_entities}
% \end{figure}

% %--- ENUMERATIONS ---
% \subsubsection{Classes Énumérations}

% \begin{figure}[H]
% \centering
% \begin{tikzpicture}[
%     enumhead/.style={rectangle, draw=gray!80, fill=gray!20, thick, text width=3.2cm, text centered, font=\small\bfseries, minimum height=0.6cm},
%     enumbody/.style={rectangle, draw=gray!80, fill=white, thick, text width=3.2cm, font=\footnotesize, align=left, minimum height=1.5cm},
%     stereotype/.style={font=\tiny\itshape, text centered}
% ]
%     % DriverStatus
%     \node[enumhead] (dstatus_h) at (0, 0) {DriverStatus};
%     \node[stereotype] at (0, 0.4) {<<enumeration>>};
%     \node[enumbody, anchor=north] (dstatus_b) at (0, -0.35) {
%         \texttt{AVAILABLE}\\
%         \texttt{ON\_ROUTE}\\
%         \texttt{OFF\_DUTY}\\
%         \texttt{ON\_BREAK}
%     };
    
%     % DriverRole
%     \node[enumhead] (drole_h) at (4, 0) {DriverRole};
%     \node[stereotype] at (4, 0.4) {<<enumeration>>};
%     \node[enumbody, anchor=north] (drole_b) at (4, -0.35) {
%         \texttt{DRIVER}\\
%         \texttt{ADMIN}\\
%         \texttt{DISPATCHER}
%     };
    
%     % VehicleType
%     \node[enumhead] (vtype_h) at (8, 0) {VehicleType};
%     \node[stereotype] at (8, 0.4) {<<enumeration>>};
%     \node[enumbody, anchor=north] (vtype_b) at (8, -0.35) {
%         \texttt{BIKE}\\
%         \texttt{CAR}\\
%         \texttt{VAN}\\
%         \texttt{TRUCK}
%     };
    
%     % RouteStatus
%     \node[enumhead, fill=successgreen!20, draw=successgreen] (rstatus_h) at (0, -3.5) {RouteStatus};
%     \node[stereotype] at (0, -3.1) {<<enumeration>>};
%     \node[enumbody, draw=successgreen, anchor=north, minimum height=2cm] (rstatus_b) at (0, -3.85) {
%         \texttt{DRAFT}\\
%         \texttt{OPTIMIZED}\\
%         \texttt{ASSIGNED}\\
%         \texttt{IN\_PROGRESS}\\
%         \texttt{COMPLETED}\\
%         \texttt{CANCELLED}
%     };
    
%     % DeliveryStatus
%     \node[enumhead, fill=warningorange!20, draw=warningorange] (delstatus_h) at (4, -3.5) {DeliveryStatus};
%     \node[stereotype] at (4, -3.1) {<<enumeration>>};
%     \node[enumbody, draw=warningorange, anchor=north, minimum height=2cm] (delstatus_b) at (4, -3.85) {
%         \texttt{PENDING}\\
%         \texttt{IN\_PROGRESS}\\
%         \texttt{DELIVERED}\\
%         \texttt{FAILED}\\
%         \texttt{CANCELLED}
%     };
    
% \end{tikzpicture}
% \caption{Diagramme de classes - Énumérations}
% \label{fig:class_diagram_enums}
% \end{figure}

% %--- VALUE OBJECTS ---
% \subsubsection{Classes Objets de Valeur}

% \begin{figure}[H]
% \centering
% \resizebox{\textwidth}{!}{
% \begin{tikzpicture}[
%     vohead/.style={rectangle, draw=primaryblue!70, fill=primaryblue!10, thick, text width=4cm, text centered, font=\small\bfseries, minimum height=0.6cm},
%     vobody/.style={rectangle, draw=primaryblue!70, fill=white, thick, text width=4cm, font=\footnotesize, align=left},
%     stereotype/.style={font=\tiny\itshape, text centered}
% ]
%     % Location
%     \node[vohead] (loc_h) at (0, 0) {Location};
%     \node[stereotype] at (0, 0.4) {<<value object>>};
%     \node[vobody, anchor=north, minimum height=1cm] (loc_b) at (0, -0.35) {
%         \texttt{- latitude: Float}\\
%         \texttt{- longitude: Float}
%     };
    
%     % Address
%     \node[vohead] (addr_h) at (5, 0) {Address};
%     \node[stereotype] at (5, 0.4) {<<value object>>};
%     \node[vobody, anchor=north, minimum height=1.5cm] (addr_b) at (5, -0.35) {
%         \texttt{- street: String}\\
%         \texttt{- city: String}\\
%         \texttt{- postal\_code: String}\\
%         \texttt{- country: String}\\
%         \texttt{- full\_address: String}
%     };
    
%     % DepotLocation
%     \node[vohead] (deploc_h) at (10, 0) {DepotLocation};
%     \node[stereotype] at (10, 0.4) {<<value object>>};
%     \node[vobody, anchor=north, minimum height=1cm] (deploc_b) at (10, -0.35) {
%         \texttt{- latitude: Float}\\
%         \texttt{- longitude: Float}\\
%         \texttt{- address: String}
%     };
    
%     % CustomerAddress
%     \node[vohead] (custaddr_h) at (15, 0) {CustomerAddress};
%     \node[stereotype] at (15, 0.4) {<<value object>>};
%     \node[vobody, anchor=north, minimum height=1.5cm] (custaddr_b) at (15, -0.35) {
%         \texttt{- street: String}\\
%         \texttt{- city: String}\\
%         \texttt{- postal\_code: String}\\
%         \texttt{- country: String}\\
%         \texttt{- latitude: Float}\\
%         \texttt{- longitude: Float}\\
%         \texttt{- notes: String}
%     };
    
%     % OptResult (RouteOptimizationResult)
%     \node[vohead, fill=successgreen!10, draw=successgreen!70] (opt_h) at (0, -3.5) {OptResult};
%     \node[stereotype] at (0, -3.1) {<<value object>>};
%     \node[vobody, draw=successgreen!70, anchor=north, minimum height=1.8cm] (opt_b) at (0, -3.85) {
%         \texttt{- optimized\_sequence: DeliveryPoint[]}\\
%         \texttt{- route\_segments: RouteSegment[]}\\
%         \texttt{- total\_distance\_km: Float}\\
%         \texttt{- total\_duration\_minutes: Float}\\
%         \texttt{- estimated\_cost: Float}\\
%         \texttt{- solver\_info: Dict}
%     };
    
%     % RouteSegment
%     \node[vohead, fill=successgreen!10, draw=successgreen!70] (seg_h) at (5, -3.5) {RouteSegment};
%     \node[stereotype] at (5, -3.1) {<<value object>>};
%     \node[vobody, draw=successgreen!70, anchor=north, minimum height=1.5cm] (seg_b) at (5, -3.85) {
%         \texttt{- from\_point\_id: String}\\
%         \texttt{- to\_point\_id: String}\\
%         \texttt{- distance\_meters: Float}\\
%         \texttt{- duration\_minutes: Float}\\
%         \texttt{- segment\_order: Integer}
%     };
    
%     % Token
%     \node[vohead] (tok_h) at (10, -3.5) {Token};
%     \node[stereotype] at (10, -3.1) {<<value object>>};
%     \node[vobody, anchor=north, minimum height=1cm] (tok_b) at (10, -3.85) {
%         \texttt{- access\_token: String}\\
%         \texttt{- token\_type: String}\\
%         \texttt{- driver\_id: String}
%     };
    
%     % DeliveryConfirmation
%     \node[vohead, fill=warningorange!10, draw=warningorange!70] (conf_h) at (15, -3.5) {DeliveryConfirmation};
%     \node[stereotype] at (15, -3.1) {<<value object>>};
%     \node[vobody, draw=warningorange!70, anchor=north, minimum height=2cm] (conf_b) at (15, -3.85) {
%         \texttt{- route\_id: String}\\
%         \texttt{- point\_id: String}\\
%         \texttt{- confirmed\_at: DateTime}\\
%         \texttt{- notes: String}\\
%         \texttt{- proof\_photo: String}\\
%         \texttt{- signature: String}\\
%         \texttt{- success: Boolean}\\
%         \texttt{- failure\_reason: String}
%     };
    
% \end{tikzpicture}
% }
% \caption{Diagramme de classes - Objets de valeur}
% \label{fig:class_diagram_valueobjects}
% \end{figure}

% %--- SERVICE CLASS ---
% \subsubsection{Classe Service Blockchain}

% \begin{figure}[H]
% \centering
% \begin{tikzpicture}[
%     svchead/.style={rectangle, draw=secondarypurple, fill=secondarypurple!20, thick, text width=6cm, text centered, font=\small\bfseries, minimum height=0.6cm},
%     svcbody/.style={rectangle, draw=secondarypurple, fill=white, thick, text width=6cm, font=\footnotesize, align=left},
%     svcmethods/.style={rectangle, draw=secondarypurple, fill=secondarypurple!5, thick, text width=6cm, font=\footnotesize, align=left},
%     stereotype/.style={font=\tiny\itshape, text centered}
% ]
%     % BlockchainService
%     \node[svchead] (bc_h) at (0, 0) {BlockchainService};
%     \node[stereotype] at (0, 0.4) {<<service>>};
%     \node[svcbody, anchor=north, minimum height=1cm] (bc_b) at (0, -0.35) {
%         \texttt{- web3: Web3}\\
%         \texttt{- contract: Contract}\\
%         \texttt{- account: Account}
%     };
%     \node[svcmethods, anchor=north, minimum height=1.8cm] (bc_m) at (0, -1.4) {
%         \texttt{+ recordRouteCreation(route\_id, route\_data)}\\
%         \texttt{+ updateRouteStatus(route\_id, status)}\\
%         \texttt{+ verifyRoute(route\_id): Boolean}\\
%         \texttt{+ getRouteFromBlockchain(route\_id)}\\
%         \texttt{+ calculateDataHash(data): String}
%     };
    
% \end{tikzpicture}
% \caption{Diagramme de classes - Service Blockchain}
% \label{fig:class_diagram_service}
% \end{figure}

% %--- SUMMARY TABLE ---
% \subsubsection{Récapitulatif des Classes}

% \begin{table}[H]
% \centering
% \begin{tabular}{|l|l|c|}
% \hline
% \textbf{Catégorie} & \textbf{Classes} & \textbf{Nombre} \\
% \hline
% \textbf{Entités} & Driver, Route, DeliveryPoint, Customer, Depot & 5 \\
% \hline
% \textbf{Objets de valeur} & Location, Address, CustomerAddress, DepotLocation, & 8 \\
%  & OptResult, RouteSegment, Token, DeliveryConfirmation & \\
% \hline
% \textbf{Énumérations} & DriverStatus, DriverRole, VehicleType, & 5 \\
%  & RouteStatus, DeliveryStatus & \\
% \hline
% \textbf{Services} & BlockchainService & 1 \\
% \hline
% \textbf{Total} & & \textbf{19} \\
% \hline
% \end{tabular}
% \caption{Récapitulatif des classes du modèle du domaine}
% \label{tab:class_summary}
% \end{table}

% %--- RELATIONSHIPS DIAGRAM ---
% \subsubsection{Diagramme des Relations UML}

% Ce diagramme montre les relations entre les entités, les énumérations (dépendance en pointillés) et les objets de valeur (composition avec losange plein).

% \begin{figure}[H]
% \centering
% \resizebox{\textwidth}{!}{
% \begin{tikzpicture}[
%     entity/.style={rectangle, draw=primaryblue, fill=primaryblue!15, thick, text width=3.5cm, text centered, font=\small\bfseries, minimum height=1.2cm, rounded corners=4pt},
%     enum/.style={rectangle, draw=gray!70, fill=gray!15, thick, text width=2.8cm, text centered, font=\footnotesize\bfseries, minimum height=1cm, rounded corners=3pt},
%     valueobj/.style={rectangle, draw=successgreen!70, fill=successgreen!10, thick, text width=2.8cm, text centered, font=\footnotesize\bfseries, minimum height=1cm, rounded corners=3pt},
%     service/.style={rectangle, draw=secondarypurple, fill=secondarypurple!15, thick, text width=3.5cm, text centered, font=\small\bfseries, minimum height=1.2cm, rounded corners=4pt},
%     stereotype/.style={font=\scriptsize\itshape},
%     % Relationship styles
%     dependency/.style={-{Stealth[length=3mm]}, dashed, thick, gray!70},
%     composition/.style={-{Diamond[fill=black, length=4mm, width=2.5mm]}, thick},
%     association/.style={-, thick},
%     uses/.style={-{Stealth[length=3mm]}, thick, secondarypurple!70},
%     mult/.style={font=\scriptsize\bfseries, fill=white, inner sep=2pt}
% ]
%     % ========== ENTITY CLASSES (CENTER ROW) ==========
%     \node[entity] (driver) at (0, 0) {Driver};
%     \node[stereotype] at (0, 0.9) {<<entity>>};
    
%     \node[entity, fill=successgreen!15, draw=successgreen] (route) at (8, 0) {Route};
%     \node[stereotype] at (8, 0.9) {<<entity>>};
    
%     \node[entity, fill=warningorange!15, draw=warningorange] (point) at (16, 0) {DeliveryPoint};
%     \node[stereotype] at (16, 0.9) {<<entity>>};
    
%     % ========== BOTTOM ENTITIES ==========
%     \node[entity, fill=secondarypurple!15, draw=secondarypurple] (customer) at (0, -8) {Customer};
%     \node[stereotype] at (0, -7.1) {<<entity>>};
    
%     \node[entity] (depot) at (8, -8) {Depot};
%     \node[stereotype] at (8, -7.1) {<<entity>>};
    
%     % ========== ENUMERATIONS (TOP ROW - spaced out) ==========
%     \node[enum] (dstatus) at (-4, 5) {DriverStatus};
%     \node[stereotype] at (-4, 5.7) {<<enumeration>>};
    
%     \node[enum] (drole) at (0, 5) {DriverRole};
%     \node[stereotype] at (0, 5.7) {<<enumeration>>};
    
%     \node[enum] (vtype) at (4, 5) {VehicleType};
%     \node[stereotype] at (4, 5.7) {<<enumeration>>};
    
%     \node[enum, fill=successgreen!15, draw=successgreen!70] (rstatus) at (8, 5) {RouteStatus};
%     \node[stereotype] at (8, 5.7) {<<enumeration>>};
    
%     \node[enum, fill=warningorange!15, draw=warningorange!70] (delstatus) at (16, 5) {DeliveryStatus};
%     \node[stereotype] at (16, 5.7) {<<enumeration>>};
    
%     % ========== VALUE OBJECTS (distributed with more space) ==========
%     % Around Route
%     \node[valueobj] (location) at (12, -4) {Location};
%     \node[stereotype] at (12, -3.3) {<<value object>>};
    
%     \node[valueobj] (address) at (16, -4) {Address};
%     \node[stereotype] at (16, -3.3) {<<value object>>};
    
%     \node[valueobj] (optresult) at (12, 3) {OptResult};
%     \node[stereotype] at (12, 3.7) {<<value object>>};
    
%     \node[valueobj] (segment) at (16, 3) {RouteSegment};
%     \node[stereotype] at (16, 3.7) {<<value object>>};
    
%     % Around Depot
%     \node[valueobj] (deploc) at (12, -8) {DepotLocation};
%     \node[stereotype] at (12, -7.3) {<<value object>>};
    
%     % Around Customer
%     \node[valueobj] (custaddr) at (4, -8) {CustomerAddress};
%     \node[stereotype] at (4, -7.3) {<<value object>>};
    
%     % Around Driver
%     \node[valueobj] (token) at (-4, -4) {Token};
%     \node[stereotype] at (-4, -3.3) {<<value object>>};
    
%     % Around DeliveryPoint
%     \node[valueobj, fill=warningorange!10, draw=warningorange!70] (confirm) at (20, -4) {DeliveryConfirmation};
%     \node[stereotype] at (20, -3.3) {<<value object>>};
    
%     % ========== SERVICE ==========
%     \node[service] (blockchain) at (20, 0) {BlockchainService};
%     \node[stereotype] at (20, 0.9) {<<service>>};
    
%     % ========== ENTITY RELATIONSHIPS (solid lines) ==========
%     \draw[association] (driver.east) -- node[mult, above, pos=0.3] {1} node[mult, above, pos=0.7] {0..*} (route.west);
%     \draw[composition] (route.east) -- node[mult, above, pos=0.3] {1} node[mult, above, pos=0.7] {5..20} (point.west);
%     \draw[association] (driver.south) -- ++(0,-2) -| node[mult, above left, pos=0.1] {1} node[mult, above right, pos=0.9] {0..*} (customer.north);
%     \draw[association] (driver.south) -- ++(0,-2) -| node[mult, above left, pos=0.1] {1} node[mult, above right, pos=0.9] {0..*} (depot.north);
%     \draw[association] (route.south) -- ++(0,-1.5) -| node[mult, above left, pos=0.1] {0..*} node[mult, above right, pos=0.9] {1} (depot.north);
    
%     % ========== ENUMERATION DEPENDENCIES (dashed arrows) ==========
%     \draw[dependency] (driver.north) -- ++(-2,1.5) -- node[mult, above, pos=0.7] {1} (dstatus.south);
%     \draw[dependency] (driver.north) -- node[mult, right, pos=0.5] {1} (drole.south);
%     \draw[dependency] (driver.north) -- ++(2,1.5) -- node[mult, above, pos=0.7] {1} (vtype.south);
%     \draw[dependency] (route.north) -- node[mult, right, pos=0.5] {1} (rstatus.south);
%     \draw[dependency] (point.north) -- node[mult, right, pos=0.5] {1} (delstatus.south);
    
%     % ========== VALUE OBJECT COMPOSITIONS (filled diamond with cardinalities) ==========
%     % Route compositions
%     \draw[composition] (route.south) -- ++(0,-1) -| node[mult, left, pos=0.8] {1} (location.north);
%     \draw[composition] (route.south) -- ++(0,-1) -| node[mult, left, pos=0.8] {0..1} (address.north);
%     \draw[composition] (route.north) -- ++(0,0.5) -| node[mult, left, pos=0.8] {0..1} (optresult.west);
    
%     % DeliveryPoint compositions
%     \draw[composition] (point.south) -- ++(0,-0.8) -| node[mult, right, pos=0.8] {1} (location.north east);
%     \draw[composition] (point.south) -- ++(0,-0.8) -| node[mult, left, pos=0.8] {1} (address.north);
%     \draw[composition] (point.south) -- ++(0,-1) -| node[mult, left, pos=0.8] {0..1} (confirm.north);
    
%     % Depot composition
%     \draw[composition] (depot.east) -- node[mult, above, pos=0.5] {1} (deploc.west);
    
%     % Customer composition
%     \draw[composition] (customer.east) -- node[mult, above, pos=0.5] {1} (custaddr.west);
    
%     % Driver returns Token
%     \draw[composition] (driver.south) -- ++(0,-1) -| node[mult, left, pos=0.8] {0..1} (token.north);
    
%     % OptResult contains RouteSegments
%     \draw[composition] (optresult.east) -- node[mult, above, pos=0.5] {1..*} (segment.west);
    
%     % Service uses Route
%     \draw[uses] (blockchain.west) -- node[font=\scriptsize, above, pos=0.5] {uses} (point.east);
    
%     % ========== LEGEND (bottom left) ==========
%     \node[font=\small\bfseries] at (-4, -10.5) {Légende :};
%     \draw[dependency] (-4, -11.2) -- (-2, -11.2) node[right, font=\scriptsize] {Dépendance (utilise enum)};
%     \draw[composition] (-4, -12) -- (-2, -12) node[right, font=\scriptsize] {Composition (possède value object)};
%     \draw[association] (-4, -12.8) -- (-2, -12.8) node[right, font=\scriptsize] {Association (référence entité)};
    
% \end{tikzpicture}
% }
% \caption{Diagramme de classes complet avec toutes les relations UML}
% \label{fig:class_diagram_relationships}
% \end{figure}

% %--- CARDINALITY TABLE ---
% \subsubsection{Tableau des Cardinalités}

% \begin{table}[H]
% \centering
% \small
% \begin{tabular}{|l|l|l|c|l|}
% \hline
% \textbf{Classe Source} & \textbf{Relation} & \textbf{Classe Cible} & \textbf{Cardinalité} & \textbf{Description} \\
% \hline
% \multicolumn{5}{|c|}{\textit{\textbf{Associations entre Entités}}} \\
% \hline
% Driver & possède & Route & 1 -- 0..* & Un driver peut avoir plusieurs routes \\
% Route & contient & DeliveryPoint & 1 -- 5..20 & Une route contient 5 à 20 points \\
% Driver & gère & Customer & 1 -- 0..* & Un driver gère ses propres clients \\
% Driver & possède & Depot & 1 -- 0..* & Un driver peut avoir plusieurs dépôts \\
% Route & part de & Depot & 0..* -- 1 & Une route part d'un seul dépôt \\
% \hline
% \multicolumn{5}{|c|}{\textit{\textbf{Compositions avec Value Objects}}} \\
% \hline
% Route & a & Location & 1 -- 1 & Position GPS du dépôt (obligatoire) \\
% Route & a & Address & 1 -- 0..1 & Adresse du dépôt (optionnelle) \\
% Route & a & OptResult & 1 -- 0..1 & Résultat optimisation (après VRP) \\
% DeliveryPoint & a & Location & 1 -- 1 & Coordonnées GPS (obligatoire) \\
% DeliveryPoint & a & Address & 1 -- 1 & Adresse de livraison (obligatoire) \\
% DeliveryPoint & a & DeliveryConfirmation & 1 -- 0..1 & Confirmation (après livraison) \\
% Customer & a & CustomerAddress & 1 -- 1 & Adresse client (obligatoire) \\
% Depot & a & DepotLocation & 1 -- 1 & Position du dépôt (obligatoire) \\
% Driver & génère & Token & 1 -- 0..1 & Token JWT (après login) \\
% OptResult & contient & RouteSegment & 1 -- 1..* & Segments de route (au moins 1) \\
% \hline
% \multicolumn{5}{|c|}{\textit{\textbf{Dépendances avec Énumérations}}} \\
% \hline
% Driver & utilise & DriverStatus & 1 -- 1 & Statut unique obligatoire \\
% Driver & utilise & DriverRole & 1 -- 1 & Rôle unique obligatoire \\
% Driver & utilise & VehicleType & 1 -- 1 & Type véhicule obligatoire \\
% Route & utilise & RouteStatus & 1 -- 1 & Statut route obligatoire \\
% DeliveryPoint & utilise & DeliveryStatus & 1 -- 1 & Statut livraison obligatoire \\
% \hline
% \end{tabular}
% \caption{Tableau complet des cardinalités du modèle de domaine}
% \label{tab:cardinalities}
% \end{table}

% \textbf{Types de relations UML utilisées :}
% \begin{itemize}
%     \item \textbf{Dépendance (- - - $\rightarrow$)} : Les entités \textit{utilisent} les énumérations comme types d'attributs. La flèche pointillée indique que l'entité dépend de l'énumération.
%     \item \textbf{Composition ($\blacklozenge$---)} : Les entités \textit{possèdent} les objets de valeur. Le losange plein indique que l'objet de valeur fait partie intégrante de l'entité et n'existe pas indépendamment.
%     \item \textbf{Association (---)} : Relations entre entités avec multiplicités (1, 0..*, 5..20, etc.).
% \end{itemize}

%---------------------------------------------------------------------
\section{Modélisation UML : Diagrammes de Séquence}
%---------------------------------------------------------------------

\subsection{Diagramme de Séquence : Authentification}

\begin{figure}[H]
\centering
\includegraphics[width=1.1\textwidth]{diagrams/authSequence.png}
\caption{Diagramme de séquence : Authentification}
\label{fig:auth_sequence_diag}
\end{figure}

% \begin{figure}[H]
% \centering
% \resizebox{\textwidth}{!}{
% \begin{tikzpicture}[
%     % Component box style (orange/yellow like the reference)
%     component/.style={rectangle, draw=warningorange!80, fill=warningorange!20, thick, minimum width=2.5cm, minimum height=0.9cm, font=\small\bfseries, rounded corners=2pt},
%     % Lifeline style
%     lifeline/.style={dashed, thick, gray!60},
%     % Message arrow style
%     message/.style={-{Stealth[length=3mm]}, thick, black},
%     % Return arrow style (dashed)
%     return/.style={-{Stealth[length=3mm]}, dashed, thick, black},
%     % Activation box style (orange)
%     activation/.style={fill=warningorange!30, draw=warningorange!80, minimum width=0.4cm},
%     % Fragment style
%     fragment/.style={rectangle, draw=black, thick},
%     fragmentlabel/.style={rectangle, fill=warningorange!40, draw=black, font=\scriptsize\bfseries, inner sep=3pt},
%     % Self message
%     selfmsg/.style={thick, black}
% ]
%     % ========== ACTOR (Stick Figure) ==========
%     \begin{scope}[shift={(0,0)}]
%         \draw[thick] (0,0.8) circle (0.25); % head
%         \draw[thick] (0,0.55) -- (0,0); % body
%         \draw[thick] (-0.3,0.4) -- (0,0.25) -- (0.3,0.4); % arms
%         \draw[thick] (0,0) -- (-0.2,-0.5); % left leg
%         \draw[thick] (0,0) -- (0.2,-0.5); % right leg
%         \node[font=\small\bfseries] at (0,-0.8) {User};
%     \end{scope}
    
%     % ========== COMPONENTS ==========
%     \node[component] (client) at (4,0.5) {Client App};
%     \node[component] (server) at (8,0.5) {Server Side};
%     \node[component] (db) at (12,0.5) {Database};
    
%     % ========== LIFELINES ==========
%     \draw[lifeline] (0,-1) -- (0,-18);
%     \draw[lifeline] (4,-0.1) -- (4,-18);
%     \draw[lifeline] (8,-0.1) -- (8,-18);
%     \draw[lifeline] (12,-0.1) -- (12,-18);
    
%     % ========== ACTIVATION BOXES ==========
%     \draw[activation] (3.8,-2) rectangle (4.2,-4);
%     \draw[activation] (7.8,-3) rectangle (8.2,-6);
%     \draw[activation] (11.8,-4) rectangle (12.2,-5.5);
%     \draw[activation] (7.8,-8) rectangle (8.2,-11);
%     \draw[activation] (3.8,-10) rectangle (4.2,-12);
    
%     % ========== MESSAGES ==========
%     % Initial request
%     \draw[message] (0,-2) -- node[above, font=\scriptsize] {Request credentials (email,password)} (3.8,-2);
%     \draw[message] (0,-3) -- node[above, font=\scriptsize] {Send Credentials (email,password)} (3.8,-3);
%     \draw[message] (4.2,-3.5) -- node[above, font=\scriptsize] {Send login request (email,password)} (7.8,-3.5);
%     \draw[message] (8.2,-4.2) -- node[above, font=\scriptsize] {Verify credentials (email,password)} (11.8,-4.2);
    
%     % Self-call at database
%     \draw[selfmsg] (12.2,-4.5) -- ++(0.8,0) -- ++(0,-0.6) -- ++(-0.8,0);
%     \node[font=\scriptsize, anchor=west] at (12.4,-4.8) {Verify credentials (email,password)};
    
%     % ========== LOOP FRAGMENT (Invalid Credentials) ==========
%     \draw[fragment] (-0.8,-6) rectangle (13,-9);
%     \node[fragmentlabel] at (-0.3,-6) {Boucle};
%     \node[font=\scriptsize, anchor=west] at (-0.6,-6.6) {[Invalid Credentials]};
    
%     \draw[return] (11.8,-6.8) -- node[above, font=\scriptsize] {Invalid credentials} (8.2,-6.8);
%     \draw[message] (0,-7.2) -- node[above, font=\scriptsize] {Request credentials (email,password)} (3.8,-7.2);
%     \draw[return] (7.8,-7.5) -- node[above, font=\scriptsize] {Invalid credentials} (4.2,-7.5);
%     \draw[message] (0,-8) -- node[above, font=\scriptsize] {Send Credentials (email,password)} (3.8,-8);
%     \draw[message] (4.2,-8.5) -- node[above, font=\scriptsize] {Send login request (email,password)} (7.8,-8.5);
    
%     % After loop - valid credentials
%     \draw[message] (8.2,-9.5) -- node[above, font=\scriptsize] {Verify credentials (email,password)} (11.8,-9.5);
%     \draw[return] (11.8,-10) -- node[above, font=\scriptsize] {Valid credentials} (8.2,-10);
    
%     % Generate JWT - self call
%     \draw[selfmsg] (8.2,-10.5) -- ++(1,0) -- ++(0,-0.6) -- ++(-1,0);
%     \node[font=\scriptsize, anchor=west] at (8.4,-10.8) {Generates JWT};
    
%     \draw[return] (7.8,-11.5) -- node[above, font=\scriptsize] {Send JWT} (4.2,-11.5);
    
%     % Store JWT - self call
%     \draw[selfmsg] (4.2,-12) -- ++(1,0) -- ++(0,-0.5) -- ++(-1,0);
%     \node[font=\scriptsize, anchor=west] at (4.4,-12.2) {Stores JWT in cookies};
    
%     % Access request
%     \draw[message] (0,-13) -- node[above, font=\scriptsize] {Makes an access request} (3.8,-13);
%     \draw[message] (4.2,-13.5) -- node[above, font=\scriptsize] {Send request + JWT} (7.8,-13.5);
    
%     % Verify JWT - self call
%     \draw[selfmsg] (8.2,-14) -- ++(1,0) -- ++(0,-0.5) -- ++(-1,0);
%     \node[font=\scriptsize, anchor=west] at (8.4,-14.2) {Verify JWT};
    
%     % ========== ALTERNATIVE FRAGMENT ==========
%     \draw[fragment] (-0.8,-15) rectangle (9,-17.5);
%     \node[fragmentlabel] at (-0.3,-15) {Alternative};
%     \node[font=\scriptsize, anchor=west] at (-0.6,-15.6) {[Valid JWT]};
%     \draw[dashed, thick] (-0.8,-16.2) -- (9,-16.2);
%     \node[font=\scriptsize, anchor=west] at (-0.6,-16.6) {[Invalid JWT]};
    
%     \draw[return] (7.8,-15.8) -- node[above, font=\scriptsize] {Allow access} (4.2,-15.8);
%     \draw[return] (7.8,-17) -- node[above, font=\scriptsize] {Reject access} (4.2,-17);
    
% \end{tikzpicture}
% }
% \caption{Diagramme de séquence - Processus d'authentification JWT}
% \label{fig:seq_auth}
% \end{figure}

\subsection{Diagramme de Séquence : Processus d'Optimisation VRP}

\begin{figure}[H]
\centering
\includegraphics[width=1\textwidth]{diagrams/routechain_sequence_vrp.pdf}
\caption{Diagramme de séquence : Processus d'Optimisation VRP}
\label{fig:vrp_sequence_diag}
\end{figure}

% \begin{figure}[H]
% \centering
% \resizebox{\textwidth}{!}{
% \begin{tikzpicture}[
%     component/.style={rectangle, draw=warningorange!80, fill=warningorange!20, thick, minimum width=2cm, minimum height=0.8cm, font=\footnotesize\bfseries, rounded corners=2pt},
%     lifeline/.style={dashed, thick, gray!60},
%     message/.style={-{Stealth[length=2.5mm]}, thick, black},
%     return/.style={-{Stealth[length=2.5mm]}, dashed, thick, black},
%     activation/.style={fill=warningorange!30, draw=warningorange!80, minimum width=0.35cm},
%     fragment/.style={rectangle, draw=black, thick},
%     fragmentlabel/.style={rectangle, fill=warningorange!40, draw=black, font=\scriptsize\bfseries, inner sep=2pt},
%     selfmsg/.style={thick, black}
% ]
%     % ========== ACTOR (Stick Figure) ==========
%     \begin{scope}[shift={(0,0)}]
%         \draw[thick] (0,0.7) circle (0.22);
%         \draw[thick] (0,0.48) -- (0,0);
%         \draw[thick] (-0.25,0.35) -- (0,0.2) -- (0.25,0.35);
%         \draw[thick] (0,0) -- (-0.18,-0.4);
%         \draw[thick] (0,0) -- (0.18,-0.4);
%         \node[font=\footnotesize\bfseries] at (0,-0.7) {Chauffeur};
%     \end{scope}
    
%     % ========== COMPONENTS ==========
%     \node[component] (front) at (3,0.4) {Frontend};
%     \node[component] (back) at (6,0.4) {Backend API};
%     \node[component] (ors) at (9,0.4) {ORS Service};
%     \node[component] (vrp) at (12,0.4) {OR-Tools};
%     \node[component] (db) at (15,0.4) {MongoDB};
%     \node[component] (bc) at (18,0.4) {Blockchain};
    
%     % ========== LIFELINES ==========
%     \draw[lifeline] (0,-1) -- (0,-16);
%     \draw[lifeline] (3,-0.1) -- (3,-16);
%     \draw[lifeline] (6,-0.1) -- (6,-16);
%     \draw[lifeline] (9,-0.1) -- (9,-16);
%     \draw[lifeline] (12,-0.1) -- (12,-16);
%     \draw[lifeline] (15,-0.1) -- (15,-16);
%     \draw[lifeline] (18,-0.1) -- (18,-16);
    
%     % ========== ACTIVATION BOXES ==========
%     \draw[activation] (2.82,-1.5) rectangle (3.18,-2.5);
%     \draw[activation] (5.82,-2) rectangle (6.18,-13);
%     \draw[activation] (8.82,-3) rectangle (9.18,-4.5);
%     \draw[activation] (11.82,-5) rectangle (12.18,-6.5);
%     \draw[activation] (14.82,-7.5) rectangle (15.18,-8.5);
%     \draw[activation] (17.82,-10) rectangle (18.18,-11.5);
    
%     % ========== MESSAGES ==========
%     % Step 1-2: User submits points
%     \draw[message] (0,-1.5) -- node[above, font=\scriptsize] {1: Submit delivery points} (2.82,-1.5);
%     \draw[message] (3.18,-2) -- node[above, font=\scriptsize] {2: optimize\_route()} (5.82,-2);
    
%     % Step 3-4: Distance matrix
%     \draw[message] (6.18,-3) -- node[above, font=\scriptsize] {3: get\_distance\_matrix()} (8.82,-3);
%     \draw[return] (8.82,-4.2) -- node[above, font=\scriptsize] {4: Return matrix[n×n]} (6.18,-4.2);
    
%     % Step 5-6: VRP optimization
%     \draw[message] (6.18,-5) -- node[above, font=\scriptsize] {5: solve\_vrp()} (11.82,-5);
%     \draw[return] (11.82,-6.2) -- node[above, font=\scriptsize] {6: optimal\_sequence} (6.18,-6.2);
    
%     % Step 7-8: Save to database
%     \draw[message] (6.18,-7.5) -- node[above, font=\scriptsize] {7: save\_route()} (14.82,-7.5);
%     \draw[return] (14.82,-8.2) -- node[above, font=\scriptsize] {8: route\_id} (6.18,-8.2);
    
%     % Step 9: Calculate hash (self-call)
%     \draw[selfmsg] (6.18,-9) -- ++(1,0) -- ++(0,-0.5) -- ++(-1,0);
%     \node[font=\scriptsize, anchor=west] at (6.4,-9.2) {9: hashing\_sha256()};
    
%     % Step 10-11: Blockchain
%     \draw[message] (6.18,-10) -- node[above, font=\scriptsize] {10: record\_on\_blockchain()} (17.82,-10);
%     \draw[return] (17.82,-11.2) -- node[above, font=\scriptsize] {11: tx\_hash, block\_number} (6.18,-11.2);
    
%     % Step 12: Update database
%     \draw[message] (6.18,-12) -- node[above, font=\scriptsize] {12: update\_blockchain\_info()} (14.82,-12);
    
%     % Step 13-14: Return to user
%     \draw[return] (5.82,-13.5) -- node[above, font=\scriptsize] {13: optimized\_route\_response} (3.18,-13.5);
%     \draw[return] (2.82,-14.5) -- node[above, font=\scriptsize] {14: Display route on map} (0,-14.5);
    
% \end{tikzpicture}
% }
% \caption{Diagramme de séquence - Processus complet d'optimisation VRP avec enregistrement blockchain}
% \label{fig:seq_vrp}
% \end{figure}

\subsection{Diagramme de Séquence : Vérification de l'Intégrité Blockchain}

\begin{figure}[H]
\centering
\includegraphics[width=1\textwidth]{diagrams/routechain_seq_blockchain.pdf}
\caption{Diagramme de séquence : Vérification de l'Intégrité Blockchain}
\label{fig:blockchain_sequence_diag}
\end{figure}
% \begin{figure}[H]
% \centering
% \resizebox{\textwidth}{!}{
% \begin{tikzpicture}[
%     component/.style={rectangle, draw=warningorange!80, fill=warningorange!20, thick, minimum width=3cm, minimum height=1cm, font=\small\bfseries, rounded corners=2pt},
%     lifeline/.style={dashed, thick, gray!60},
%     message/.style={-{Stealth[length=3mm]}, thick, black},
%     return/.style={-{Stealth[length=3mm]}, dashed, thick, black},
%     activation/.style={fill=warningorange!30, draw=warningorange!80, minimum width=0.45cm},
%     fragment/.style={rectangle, draw=black, thick},
%     fragmentlabel/.style={rectangle, fill=warningorange!40, draw=black, font=\scriptsize\bfseries, inner sep=4pt},
%     selfmsg/.style={thick, black}
% ]
%     % ========== ACTOR (Stick Figure) ==========
%     \begin{scope}[shift={(0,0)}]
%         \draw[thick] (0,0.9) circle (0.3);
%         \draw[thick] (0,0.6) -- (0,-0.1);
%         \draw[thick] (-0.35,0.45) -- (0,0.25) -- (0.35,0.45);
%         \draw[thick] (0,-0.1) -- (-0.25,-0.7);
%         \draw[thick] (0,-0.1) -- (0.25,-0.7);
%         \node[font=\small\bfseries] at (0,-1.1) {User};
%     \end{scope}
    
%     % ========== COMPONENTS (more spaced) ==========
%     \node[component] (front) at (5,0.5) {Frontend};
%     \node[component] (back) at (11,0.5) {Backend API};
%     \node[component] (db) at (17,0.5) {MongoDB};
    
%     % ========== LIFELINES (longer) ==========
%     \draw[lifeline] (0,-1.5) -- (0,-20);
%     \draw[lifeline] (5,-0.1) -- (5,-20);
%     \draw[lifeline] (11,-0.1) -- (11,-20);
%     \draw[lifeline] (17,-0.1) -- (17,-20);
    
%     % ========== ACTIVATION BOXES ==========
%     \draw[activation] (4.77,-2) rectangle (5.23,-3.5);
%     \draw[activation] (10.77,-3.5) rectangle (11.23,-13);
%     \draw[activation] (16.77,-5) rectangle (17.23,-7);
    
%     % ========== MESSAGES ==========
%     % Step 1: User clicks verify
%     \draw[message] (0,-2.5) -- node[above, font=\scriptsize] {1: click\_verify\_button()} (4.77,-2.5);
    
%     % Step 2: Frontend calls backend
%     \draw[message] (5.23,-3.5) -- node[above, font=\scriptsize] {2: verify\_route\_integrity(route\_id)} (10.77,-3.5);
    
%     % Step 3-4: Get route from database
%     \draw[message] (11.23,-5) -- node[above, font=\scriptsize] {3: get\_route(route\_id)} (16.77,-5);
%     \draw[return] (16.77,-6.5) -- node[above, font=\scriptsize] {4: route\_document (with stored\_hash)} (11.23,-6.5);
    
%     % Step 5: Extract stored hash (self-call)
%     \draw[selfmsg] (11.23,-8) -- ++(1.5,0) -- ++(0,-0.7) -- ++(-1.5,0);
%     \node[font=\scriptsize, anchor=west] at (11.5,-8.3) {5: extract\_stored\_hash(route\_data)};
    
%     % Step 6: Calculate current hash (self-call)
%     \draw[selfmsg] (11.23,-9.5) -- ++(1.5,0) -- ++(0,-0.7) -- ++(-1.5,0);
%     \node[font=\scriptsize, anchor=west] at (11.5,-9.8) {6: calculate\_hash\_sha256(route\_data)};
    
%     % Step 7: Compare hashes (self-call)
%     \draw[selfmsg] (11.23,-11) -- ++(1.5,0) -- ++(0,-0.7) -- ++(-1.5,0);
%     \node[font=\scriptsize, anchor=west] at (11.5,-11.3) {7: compare\_hashes(stored, current)};
    
%     % ========== ALTERNATIVE FRAGMENT ==========
%     \draw[fragment] (-0.8,-14) rectangle (16,-19);
%     \node[fragmentlabel] at (-0.2,-14) {Alternative};
%     \node[font=\scriptsize\bfseries, anchor=west] at (-0.5,-14.8) {[stored\_hash == current\_hash]};
%     \draw[dashed, thick] (-0.8,-16.3) -- (16,-16.3);
%     \node[font=\scriptsize\bfseries, anchor=west] at (-0.5,-17) {[stored\_hash != current\_hash]};
    
%     % Valid case - green arrows
%     \draw[return, green!60!black] (10.77,-15.3) -- node[above, font=\scriptsize, fill=green!15, rounded corners=2pt] {8a: \{verified: true, integrity: "VALID"\}} (5.23,-15.3);
%     \draw[return, green!60!black] (4.77,-15.8) -- node[above, font=\scriptsize] {9a: Display $\checkmark$ Data Integrity Verified} (0,-15.8);
    
%     % Invalid case - red arrows
%     \draw[return, red!70] (10.77,-17.8) -- node[above, font=\scriptsize, fill=red!15, rounded corners=2pt] {8b: \{verified: false, integrity: "TAMPERED"\}} (5.23,-17.8);
%     \draw[return, red!70] (4.77,-18.3) -- node[above, font=\scriptsize] {9b: Display $\times$ Data Has Been Modified!} (0,-18.3);
    
% \end{tikzpicture}
% }
% \caption{Diagramme de séquence - Vérification de l'intégrité via blockchain}
% \label{fig:seq_verify}
% \end{figure}

%---------------------------------------------------------------------
\section{Conception du Schéma de Base de Données}
%---------------------------------------------------------------------

\subsection{Collections MongoDB}

RouteChain utilise MongoDB, une base de données NoSQL orientée documents. Les principales collections sont décrites ci-dessous.

\begin{table}[H]
\centering
\caption{Structure de la collection \texttt{drivers}}
\begin{tabular}{|l|l|l|}
\hline
\textbf{Champ} & \textbf{Type} & \textbf{Description} \\
\hline
\_id & ObjectId & Identifiant MongoDB \\
driver\_id & String & Identifiant unique (DRV\_YYYYMMDD\_HHMMSS) \\
email & String & Adresse email (unique) \\
password\_hash & String & Mot de passe hashé (bcrypt) \\
first\_name & String & Prénom \\
last\_name & String & Nom de famille \\
role & String & Rôle (admin ou driver) \\
created\_at & DateTime & Date de création \\
\hline
\end{tabular}
\end{table}

\begin{table}[H]
\centering
\caption{Structure de la collection \texttt{routes}}
\begin{tabular}{|l|l|l|}
\hline
\textbf{Champ} & \textbf{Type} & \textbf{Description} \\
\hline
\_id & ObjectId & Identifiant MongoDB \\
route\_id & String & Identifiant unique (ROUTE\_YYYYMMDD\_HHMMSS) \\
route\_name & String & Nom de la tournée \\
driver\_id & String & Référence au chauffeur \\
status & String & Statut (optimized, in\_progress, completed) \\
depot\_location & Object & Coordonnées du dépôt \\
delivery\_points & Array & Liste des points de livraison \\
optimization\_result & Object & Résultats de l'optimisation \\
blockchain\_tx\_hash & String & Hash de la transaction blockchain \\
blockchain\_block & Integer & Numéro du bloc \\
data\_hash & String & Hash SHA-256 des données \\
created\_at & DateTime & Date de création \\
completed\_at & DateTime & Date de complétion \\
\hline
\end{tabular}
\end{table}

%---------------------------------------------------------------------
\section{Architecture du Smart Contract}
%---------------------------------------------------------------------

\subsection{Structure du Contrat RouteRegistry}

Le smart contract \texttt{RouteRegistry} est déployé sur la blockchain Ethereum (Ganache en développement) et assure l'immuabilité des enregistrements de tournées.

\begin{figure}[H]
\centering
\begin{tikzpicture}[
    contract/.style={rectangle, draw=secondarypurple!80, fill=secondarypurple!5, very thick, minimum width=12cm, rounded corners=8pt},
    section/.style={rectangle, draw=gray!60, fill=white, thick, minimum width=11cm, rounded corners=4pt, inner sep=8pt, font=\footnotesize, align=left},
    title/.style={font=\large\bfseries, secondarypurple},
    keyword/.style={font=\scriptsize\bfseries, color=primaryblue!80},
    type/.style={font=\scriptsize\ttfamily, color=gray!80}
]
    % Contract box - reduced height
    \node[contract, minimum height=11.5cm] (c) at (0,-0.5) {};
    \node[title] at (0,4.5) {<<contract>> RouteRegistry};
    % \node[font=\scriptsize\itshape, gray] at (0,4.0) {Solidity 0.8.0};
    
    % Struct - moved up
    \node[section, fill=primaryblue!5, draw=primaryblue!60] at (0,2.5) {
        \textbf{Struct RouteRecord}\\[4pt]
        \texttt{\textcolor{primaryblue}{string} routeId}\\
        \texttt{\textcolor{primaryblue}{bytes32} dataHash}\\
        \texttt{\textcolor{primaryblue}{uint256} timestamp, totalDistance, deliveryCount}\\
        \texttt{\textcolor{primaryblue}{address} driver}\\
        \texttt{\textcolor{primaryblue}{string} status \textcolor{gray}{\textit{// optimized|in\_progress|completed}}}\\
        \texttt{\textcolor{primaryblue}{bool} exists}
    };
    
    % State Variables - moved up
    \node[section] at (0,-0.03) {
        \textbf{State Variables}\\[4pt]
        \texttt{\textcolor{primaryblue}{mapping}(string => RouteRecord) \textbf{routes}} \hfill \textcolor{gray}{\textit{// Storage}}\\
        \texttt{\textcolor{primaryblue}{string[]} \textbf{routeIds}} \hfill \textcolor{gray}{\textit{// Index}}
    };
    
    % Functions - moved up
    \node[section, fill=successgreen!5, draw=successgreen!60] at (0,-2.5) {
        \textbf{Functions}\\[4pt]
        \texttt{+ createRoute(...) \hfill \textcolor{successgreen}{[external]}}\\
        \texttt{+ updateRoute(...) \hfill \textcolor{successgreen}{[external]}}\\
        \texttt{+ verifyRoute(...) : bool \hfill \textcolor{primaryblue}{[view]}}\\
        \texttt{+ getRoute(...) : RouteRecord \hfill \textcolor{primaryblue}{[view]}}\\
        \texttt{+ getRouteCount() : uint256 \hfill \textcolor{primaryblue}{[view]}}\\
        \texttt{+ getRouteIdByIndex(...) : string \hfill \textcolor{primaryblue}{[view]}}
    };
    
    % Events - moved up
    \node[section, fill=warningorange!5, draw=warningorange!60] at (0,-5.2) {
        \textbf{Events}\\[4pt]
        \texttt{* RouteCreated(\textbf{indexed} routeId, dataHash, ...)}\\
        \texttt{* RouteUpdated(\textbf{indexed} routeId, newDataHash, ...)}\\
        \texttt{* RouteCompleted(\textbf{indexed} routeId, finalHash, ...)}
    };
    
\end{tikzpicture}
\caption{Structure du smart contract RouteRegistry}
\label{fig:smart_contract}
\end{figure}

\subsubsection{Estimation du Gas Consommé}

Le tableau suivant présente l'estimation du coût en gas pour chaque fonction du smart contract :

\begin{table}[H]
\centering
\begin{tabular}{|l|c|c|c|c|}
\hline
\textbf{Fonction Solidity} & \textbf{Type} & \textbf{Gas estimé} & \textbf{Coût ETH (20 gwei)} & \textbf{Coût en MAD} \\
\hline
\texttt{createRoute} & write & $\sim$80 000 gas & 0.0016 ETH & $\sim$45 MAD \\
\hline
\texttt{updateRoute} & write & $\sim$50 000 gas & 0.001 ETH & $\sim$28 MAD \\
\hline
\texttt{verifyRoute} & view & 0 & 0 & 0 \\
\hline
\texttt{getRoute} & view & 0 & 0 & 0 \\
\hline
\texttt{getRouteCount} & view & 0 & 0 & 0 \\
\hline
\end{tabular}
\caption{Estimation du gas consommé par fonction}
\label{tab:gas_estimation}
\end{table}

\textbf{Note :} Les fonctions de type \texttt{view} ne consomment pas de gas car elles ne modifient pas l'état de la blockchain.

\textbf{Explication :} Le \textit{gas} est l'unité de mesure du coût computationnel sur Ethereum. Chaque opération (écriture, calcul, stockage) consomme une quantité de gas. Les fonctions d'écriture (\texttt{write}) modifient l'état de la blockchain et nécessitent une transaction payante, tandis que les fonctions de lecture (\texttt{view}) sont gratuites car elles interrogent uniquement les données existantes. Le coût en MAD est estimé sur la base du taux de change ETH/MAD au moment du développement.

%---------------------------------------------------------------------
\section{Conclusion}
%---------------------------------------------------------------------

Ce chapitre a présenté la conception complète du système RouteChain à travers différents niveaux d'abstraction. L'architecture trois-tiers garantit une séparation claire des responsabilités, tandis que les diagrammes UML offrent une vision globale et détaillée du fonctionnement du système.

Les diagrammes de séquence mettent particulièrement en évidence les deux aspects innovants de l'application : le processus d'optimisation VRP avec ses multiples composants et l'interaction avec la blockchain pour la traçabilité des données.

Le chapitre suivant sera consacré à la réalisation technique, présentant les technologies utilisées et les interfaces développées.



    