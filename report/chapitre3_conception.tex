%=====================================================================
%            CHAPITRE 3 : CONCEPTION DU SYSTÈME
%=====================================================================

\chapter{Conception du Système}

%---------------------------------------------------------------------
\section{Introduction}
%---------------------------------------------------------------------

Ce chapitre présente la conception détaillée du système RouteChain à travers différents niveaux d'abstraction. Nous commencerons par une vue d'ensemble de l'architecture globale, puis nous utiliserons le langage de modélisation UML pour représenter les différents aspects du système : cas d'utilisation, structure statique et comportement dynamique. Nous terminerons par la conception du schéma de base de données et l'architecture du smart contract.

%---------------------------------------------------------------------
\section{Architecture Globale du Système}
%---------------------------------------------------------------------

\subsection{Architecture Trois-Tiers}

RouteChain adopte une architecture trois-tiers classique, séparant clairement les responsabilités entre la présentation, la logique métier et la persistance des données.

\begin{figure}[H]
\centering
\begin{tikzpicture}[
    tier/.style={rectangle, draw=primaryblue, fill=primaryblue!15, thick, minimum width=13cm, minimum height=2.2cm, text centered, rounded corners=8pt, drop shadow},
    component/.style={rectangle, draw=secondarypurple, fill=white, thick, minimum width=3.2cm, minimum height=1.1cm, text centered, rounded corners=5pt, font=\small\bfseries, drop shadow={opacity=0.3}},
    arrow/.style={-{Stealth[length=4mm, width=3mm]}, line width=1.5pt, primaryblue}
]
    % Tiers with gradient effect
    \node[tier] (presentation) at (0,7) {};
    \node[font=\large\bfseries, primaryblue] at (0,8.2) {Couche Présentation};
    
    \node[tier, fill=successgreen!15, draw=successgreen] (logique) at (0,3.5) {};
    \node[font=\large\bfseries, successgreen] at (0,4.7) {Couche Logique Métier};
    
    \node[tier, fill=warningorange!15, draw=warningorange] (donnees) at (0,0) {};
    \node[font=\large\bfseries, warningorange] at (0,1.2) {Couche Données};
    
    % Components - Presentation
    \node[component] at (-4,6.5) {React 18};
    \node[component] at (0,6.5) {Leaflet Maps};
    \node[component] at (4,6.5) {Tailwind CSS};
    
    % Components - Logic
    \node[component, draw=successgreen] at (-4,3) {FastAPI};
    \node[component, draw=successgreen] at (0,3) {Google OR-Tools};
    \node[component, draw=successgreen] at (4,3) {Web3.py};
    
    % Components - Data
    \node[component, draw=warningorange] at (-2.5,-0.5) {MongoDB Atlas};
    \node[component, draw=warningorange] at (2.5,-0.5) {Ethereum};
    
    % Arrows
    \draw[arrow] (0,5.8) -- (0,4.9);
    \draw[arrow] (0,2.3) -- (0,1.4);
    
\end{tikzpicture}
\caption{Architecture trois-tiers de RouteChain}
\label{fig:architecture_3tiers}
\end{figure}

\subsection{Diagramme d'Interaction des Composants}

\begin{figure}[H]
\centering
\begin{tikzpicture}[
    box/.style={rectangle, draw=primaryblue, fill=primaryblue!10, thick, minimum width=2.8cm, minimum height=1.4cm, text centered, rounded corners=6pt, font=\small\bfseries, align=center, drop shadow={opacity=0.3}},
    ext/.style={rectangle, draw=warningorange, fill=warningorange!10, thick, minimum width=2.8cm, minimum height=1.4cm, text centered, rounded corners=6pt, font=\small\bfseries, align=center, drop shadow={opacity=0.3}},
    arrow/.style={-{Stealth[length=3mm]}, line width=1.2pt, gray!70},
    label/.style={font=\tiny\bfseries, fill=white, inner sep=2pt}
]
    % Internal components
    \node[box] (frontend) at (0,0) {Frontend\\React};
    \node[box] (backend) at (4.5,0) {Backend\\FastAPI};
    \node[box, fill=successgreen!10, draw=successgreen] (mongodb) at (9,2) {MongoDB\\Atlas};
    \node[box, fill=secondarypurple!10, draw=secondarypurple] (blockchain) at (9,-2) {Ganache\\Ethereum};
    
    % External services
    \node[ext] (ors) at (4.5,-3.5) {OpenRoute\\Service};
    \node[ext] (nominatim) at (0,-3.5) {Nominatim\\Geocoding};
    
    % Arrows with labels
    \draw[arrow] (frontend) -- node[label, above] {REST API} (backend);
    \draw[arrow] (backend) -- node[label, sloped, above] {Motor} (mongodb);
    \draw[arrow] (backend) -- node[label, sloped, above] {Web3} (blockchain);
    \draw[arrow] (backend) -- node[label, right] {HTTP} (ors);
    \draw[arrow] (frontend) -- node[label, left] {HTTP} (nominatim);
    
\end{tikzpicture}
\caption{Diagramme d'interaction des composants}
\label{fig:composants}
\end{figure}

%---------------------------------------------------------------------
\section{Modélisation UML : Diagrammes de Cas d'Utilisation}
%---------------------------------------------------------------------

% Define stick figure command
\newcommand{\stickman}[1]{%
    \begin{scope}[shift={#1}, scale=0.4]
        \draw[thick] (0,2) circle (0.4); % head
        \draw[thick] (0,1.6) -- (0,0.5); % body
        \draw[thick] (-0.5,1.3) -- (0,1.1) -- (0.5,1.3); % arms
        \draw[thick] (0,0.5) -- (-0.4,0) -- (-0.5,-0.5); % left leg
        \draw[thick] (0,0.5) -- (0.4,0) -- (0.5,-0.5); % right leg
    \end{scope}
}

\subsection{Diagramme de Cas d'Utilisation Global}

\begin{figure}[H]
\centering
\begin{tikzpicture}[
    usecase/.style={ellipse, draw=primaryblue, fill=primaryblue!8, thick, minimum width=4cm, minimum height=1.2cm, text centered, font=\small, drop shadow={opacity=0.2}},
    system/.style={rectangle, draw=primaryblue!50, fill=blue!3, thick, rounded corners=10pt},
    actorbox/.style={rectangle, draw=none, fill=none}
]
    % System boundary
    \draw[system] (1,-5.5) rectangle (10,3.5);
    \node[font=\Large\bfseries, primaryblue, anchor=north west] at (1.3,3.3) {RouteChain};
    
    % Actor - Chauffeur
    \stickman{(-1, 1)}
    \node[font=\small\bfseries] at (-1, -0.5) {Chauffeur};
    
    % Actor - Admin
    \stickman{(-1, -4)}
    \node[font=\small\bfseries] at (-1, -5.5) {Admin};
    
    % Use cases
    \node[usecase] (auth) at (5.5, 3) {S'authentifier};
    \node[usecase] (create) at (5.5, 1.5) {Créer une tournée};
    \node[usecase] (optimize) at (5.5, 0) {Optimiser itinéraire};
    \node[usecase] (confirm) at (5.5, -1.5) {Confirmer livraison};
    \node[usecase, fill=secondarypurple!8, draw=secondarypurple] (verify) at (5.5, -3) {Vérifier blockchain};
    \node[usecase, fill=successgreen!8, draw=successgreen] (stats) at (5.5, -4.5) {Consulter analytics};
    
    % Connections - Driver (blue)
    \draw[->, thick, primaryblue!70] (-0.2, 1.4) -- (auth.west);
    \draw[->, thick, primaryblue!70] (-0.2, 1) -- (create.west);
    \draw[->, thick, primaryblue!70] (-0.2, 0.6) -- (optimize.west);
    \draw[->, thick, primaryblue!70] (-0.2, 0.2) -- (confirm.west);
    \draw[->, thick, secondarypurple!70] (-0.2, -0.2) -- (verify.west);
    
    % Connections - Admin (orange)
    \draw[->, thick, warningorange!70] (-0.2, -3.6) -- (auth.west);
    \draw[->, thick, warningorange!70] (-0.2, -4) -- (stats.west);
    \draw[->, thick, warningorange!70] (-0.2, -4.4) -- (verify.west);
    
\end{tikzpicture}
\caption{Diagramme de cas d'utilisation global du système RouteChain}
\label{fig:uc_global}
\end{figure}

\subsection{Diagramme de Cas d'Utilisation : Sous-système Chauffeur}

\begin{figure}[H]
\centering
\begin{tikzpicture}[
    usecase/.style={ellipse, draw=primaryblue, fill=primaryblue!8, thick, minimum width=3.8cm, minimum height=1cm, text centered, font=\footnotesize, drop shadow={opacity=0.2}},
    system/.style={rectangle, draw=primaryblue!50, fill=blue!3, thick, rounded corners=10pt},
    include/.style={->, dashed, thick, gray}
]
    % System boundary
    \draw[system] (1,-5) rectangle (9,2.5);
    \node[font=\large\bfseries, primaryblue, anchor=north west] at (1.3,2.3) {Gestion des Tournées};
    
    % Actor
    \stickman{(-1, -1)}
    \node[font=\small\bfseries] at (-1, -2.5) {Chauffeur};
    
    % Use cases
    \node[usecase] (new) at (5, 2) {Créer nouvelle tournée};
    \node[usecase] (add) at (5, 0.7) {Ajouter points de livraison};
    \node[usecase] (opt) at (5, -0.6) {Lancer optimisation VRP};
    \node[usecase] (view) at (5, -1.9) {Visualiser sur carte};
    \node[usecase] (start) at (5, -3.2) {Démarrer la tournée};
    \node[usecase, fill=successgreen!8, draw=successgreen] (deliver) at (5, -4.5) {Confirmer livraison};
    
    % Connections
    \draw[->, thick, primaryblue!70] (-0.2, -0.6) -- (new.west);
    \draw[->, thick, primaryblue!70] (-0.2, -0.8) -- (add.west);
    \draw[->, thick, primaryblue!70] (-0.2, -1) -- (opt.west);
    \draw[->, thick, primaryblue!70] (-0.2, -1.2) -- (view.west);
    \draw[->, thick, primaryblue!70] (-0.2, -1.4) -- (start.west);
    \draw[->, thick, successgreen!70] (-0.2, -1.6) -- (deliver.west);
    
    % Include relationships
    \draw[include] (new.south) -- node[font=\tiny, right, pos=0.5] {<<include>>} (add.north);
    \draw[include] (add.south) -- node[font=\tiny, right, pos=0.5] {<<include>>} (opt.north);
    
\end{tikzpicture}
\caption{Diagramme de cas d'utilisation - Sous-système Chauffeur}
\label{fig:uc_driver}
\end{figure}

\subsection{Diagramme de Cas d'Utilisation : Sous-système Administrateur}

\begin{figure}[H]
\centering
\begin{tikzpicture}[
    usecase/.style={ellipse, draw=secondarypurple, fill=secondarypurple!8, thick, minimum width=3.8cm, minimum height=1cm, text centered, font=\footnotesize, drop shadow={opacity=0.2}},
    system/.style={rectangle, draw=secondarypurple!50, fill=purple!3, thick, rounded corners=10pt}
]
    % System boundary
    \draw[system] (1,-3.5) rectangle (9,2);
    \node[font=\large\bfseries, secondarypurple, anchor=north west] at (1.3,1.8) {Administration};
    
    % Actor
    \stickman{(-1, -0.5)}
    \node[font=\small\bfseries] at (-1, -2) {Admin};
    
    % Use cases
    \node[usecase] (drivers) at (5, 1.5) {Lister tous les chauffeurs};
    \node[usecase] (role) at (5, 0.3) {Modifier rôle utilisateur};
    \node[usecase] (routes) at (5, -0.9) {Voir toutes les tournées};
    \node[usecase, fill=successgreen!8, draw=successgreen] (analytics) at (5, -2.1) {Accéder aux analytics};
    \node[usecase] (manage) at (5, -3.3) {Gérer clients et dépôts};
    
    % Connections
    \draw[->, thick, secondarypurple!70] (-0.2, -0.1) -- (drivers.west);
    \draw[->, thick, secondarypurple!70] (-0.2, -0.3) -- (role.west);
    \draw[->, thick, secondarypurple!70] (-0.2, -0.5) -- (routes.west);
    \draw[->, thick, successgreen!70] (-0.2, -0.7) -- (analytics.west);
    \draw[->, thick, secondarypurple!70] (-0.2, -0.9) -- (manage.west);
    
\end{tikzpicture}
\caption{Diagramme de cas d'utilisation - Sous-système Administrateur}
\label{fig:uc_admin}
\end{figure}

%---------------------------------------------------------------------
\section{Modélisation UML : Diagramme de Classes}
%---------------------------------------------------------------------

\subsection{Modèle du Domaine}

\begin{figure}[H]
\centering
\resizebox{\textwidth}{!}{
\begin{tikzpicture}[
    classhead/.style={rectangle, draw=primaryblue, fill=primaryblue!20, thick, text width=4.2cm, text centered, font=\small\bfseries, minimum height=0.7cm},
    classbody/.style={rectangle, draw=primaryblue, fill=white, thick, text width=4.2cm, font=\footnotesize, minimum height=2cm, align=left},
    arrow/.style={->, thick, gray},
    comp/.style={-{Diamond[fill=black, length=3mm, width=2mm]}, thick},
    assoc/.style={-, thick},
    mult/.style={font=\tiny\bfseries, fill=white, inner sep=1pt}
]
    % Driver class
    \node[classhead] (driver_h) at (0, 5) {Driver};
    \node[classbody, anchor=north] (driver_b) at (0, 4.65) {
        \texttt{- driver\_id: String}\\
        \texttt{- email: String}\\
        \texttt{- password\_hash: String}\\
        \texttt{- first\_name: String}\\
        \texttt{- last\_name: String}\\
        \texttt{- role: String}\\
        \texttt{- created\_at: DateTime}
    };
    
    % Route class
    \node[classhead, fill=successgreen!20, draw=successgreen] (route_h) at (6, 5) {Route};
    \node[classbody, draw=successgreen, anchor=north] (route_b) at (6, 4.65) {
        \texttt{- route\_id: String}\\
        \texttt{- route\_name: String}\\
        \texttt{- status: String}\\
        \texttt{- total\_distance: Float}\\
        \texttt{- blockchain\_tx: String}\\
        \texttt{- data\_hash: String}\\
        \texttt{- created\_at: DateTime}
    };
    
    % DeliveryPoint class
    \node[classhead, fill=warningorange!20, draw=warningorange] (point_h) at (12, 5) {DeliveryPoint};
    \node[classbody, draw=warningorange, anchor=north] (point_b) at (12, 4.65) {
        \texttt{- point\_id: String}\\
        \texttt{- customer\_name: String}\\
        \texttt{- address: String}\\
        \texttt{- latitude: Float}\\
        \texttt{- longitude: Float}\\
        \texttt{- status: String}\\
        \texttt{- sequence\_order: Int}
    };
    
    % Customer class
    \node[classhead, fill=secondarypurple!20, draw=secondarypurple] (cust_h) at (0, -1) {Customer};
    \node[classbody, draw=secondarypurple, anchor=north] (cust_b) at (0, -1.35) {
        \texttt{- customer\_id: String}\\
        \texttt{- name: String}\\
        \texttt{- email: String}\\
        \texttt{- phone: String}\\
        \texttt{- address: String}
    };
    
    % Depot class
    \node[classhead] (depot_h) at (6, -1) {Depot};
    \node[classbody, anchor=north] (depot_b) at (6, -1.35) {
        \texttt{- depot\_id: String}\\
        \texttt{- name: String}\\
        \texttt{- address: String}\\
        \texttt{- latitude: Float}\\
        \texttt{- longitude: Float}
    };
    
    % Relationships with multiplicities
    \draw[assoc] (driver_b.east) -- node[mult, above, pos=0.2] {1} node[mult, above, pos=0.8] {*} node[mult, below, pos=0.5] {possède} (route_b.west);
    \draw[comp] (route_b.east) -- node[mult, above, pos=0.2] {1} node[mult, above, pos=0.8] {*} node[mult, below, pos=0.5] {contient} (point_b.west);
    \draw[assoc] (route_b.south) -- ++(0,-0.8) -| node[mult, above left, pos=0.25] {*} node[mult, below, pos=0.5] {part de} (depot_h.north) node[mult, above right, pos=0.95] {1};
    \draw[assoc] (point_b.south) -- ++(0,-1.5) -| node[mult, above left, pos=0.25] {*} node[mult, below, pos=0.5] {référence} (cust_h.north) node[mult, above right, pos=0.95] {0..1};
    
\end{tikzpicture}
}
\caption{Diagramme de classes - Modèle du domaine}
\label{fig:class_diagram}
\end{figure}

%---------------------------------------------------------------------
\section{Modélisation UML : Diagrammes de Séquence}
%---------------------------------------------------------------------

\subsection{Diagramme de Séquence : Authentification}

\begin{figure}[H]
\centering
\begin{tikzpicture}[
    actor/.style={rectangle, draw=primaryblue, fill=primaryblue!15, minimum width=2cm, minimum height=0.8cm, font=\footnotesize\bfseries, rounded corners=3pt, drop shadow={opacity=0.2}},
    lifeline/.style={dashed, thick, gray!60},
    message/.style={-{Stealth[length=2.5mm]}, thick, primaryblue},
    return/.style={-{Stealth[length=2.5mm]}, dashed, thick, successgreen},
    activation/.style={fill=primaryblue!20, draw=primaryblue},
    note/.style={rectangle, draw=warningorange, fill=warningorange!10, font=\tiny, text width=2cm, rounded corners=2pt}
]
    % Actors/Objects
    \node[actor] (user) at (0,0) {Utilisateur};
    \node[actor] (front) at (3.5,0) {Frontend};
    \node[actor, fill=successgreen!15, draw=successgreen] (back) at (7,0) {Backend};
    \node[actor, fill=warningorange!15, draw=warningorange] (db) at (10.5,0) {MongoDB};
    
    % Lifelines
    \draw[lifeline] (0,-0.4) -- (0,-8);
    \draw[lifeline] (3.5,-0.4) -- (3.5,-8);
    \draw[lifeline] (7,-0.4) -- (7,-8);
    \draw[lifeline] (10.5,-0.4) -- (10.5,-8);
    
    % Activation boxes
    \draw[activation] (6.85,-2) rectangle (7.15,-5.5);
    
    % Messages
    \draw[message] (0,-1) -- node[above, font=\tiny\bfseries] {1: saisie credentials} (3.5,-1);
    \draw[message] (3.5,-2) -- node[above, font=\tiny\bfseries] {2: POST /auth/login} (7,-2);
    \draw[message] (7,-2.8) -- node[above, font=\tiny\bfseries] {3: find\_one(email)} (10.5,-2.8);
    \draw[return] (10.5,-3.5) -- node[above, font=\tiny\bfseries] {4: driver\_document} (7,-3.5);
    
    % Note for processing
    \node[note] at (9,-4.5) {verify\_password\\generate\_jwt()};
    
    \draw[return] (7,-5.8) -- node[above, font=\tiny\bfseries] {5: \{token, user\}} (3.5,-5.8);
    \draw[message] (3.5,-6.5) -- node[right, font=\tiny\bfseries] {6: store token} (3.5,-6.8);
    \draw[return] (3.5,-7.2) -- node[above, font=\tiny\bfseries] {7: redirect dashboard} (0,-7.2);
    
\end{tikzpicture}
\caption{Diagramme de séquence - Processus d'authentification JWT}
\label{fig:seq_auth}
\end{figure}

\subsection{Diagramme de Séquence : Processus d'Optimisation VRP}

\begin{figure}[H]
\centering
\resizebox{\textwidth}{!}{
\begin{tikzpicture}[
    actor/.style={rectangle, draw=primaryblue, fill=primaryblue!15, minimum width=1.6cm, minimum height=0.7cm, font=\tiny\bfseries, rounded corners=3pt},
    lifeline/.style={dashed, thick, gray!50},
    message/.style={-{Stealth[length=2mm]}, thick, primaryblue},
    return/.style={-{Stealth[length=2mm]}, dashed, thick, successgreen},
    activation/.style={fill=primaryblue!20, draw=primaryblue}
]
    % Actors/Objects
    \node[actor] (user) at (0,0) {Chauffeur};
    \node[actor] (front) at (2.5,0) {Frontend};
    \node[actor, fill=successgreen!15, draw=successgreen] (back) at (5,0) {Backend};
    \node[actor, fill=warningorange!15, draw=warningorange] (ors) at (7.5,0) {ORS API};
    \node[actor, fill=secondarypurple!15, draw=secondarypurple] (vrp) at (10,0) {OR-Tools};
    \node[actor] (db) at (12.5,0) {MongoDB};
    \node[actor, fill=gray!20, draw=gray] (bc) at (15,0) {Blockchain};
    
    % Lifelines
    \foreach \x in {0,2.5,5,7.5,10,12.5,15} {
        \draw[lifeline] (\x,-0.35) -- (\x,-11);
    }
    
    % Activation box for backend
    \draw[activation] (4.85,-1.5) rectangle (5.15,-9.5);
    
    % Messages
    \draw[message] (0,-0.8) -- node[above, font=\tiny\bfseries] {1: submit points} (2.5,-0.8);
    \draw[message] (2.5,-1.5) -- node[above, font=\tiny\bfseries] {2: POST /routes/optimize} (5,-1.5);
    
    \draw[message] (5,-2.2) -- node[above, font=\tiny\bfseries] {3: distance\_matrix()} (7.5,-2.2);
    \draw[return] (7.5,-3) -- node[above, font=\tiny\bfseries] {4: matrix[n×n]} (5,-3);
    
    \draw[message] (5,-3.8) -- node[above, font=\tiny\bfseries] {5: solve\_vrp(matrix)} (10,-3.8);
    \draw[return] (10,-4.6) -- node[above, font=\tiny\bfseries] {6: optimal\_sequence} (5,-4.6);
    
    \draw[message] (5,-5.4) -- node[above, font=\tiny\bfseries] {7: insert\_one(route)} (12.5,-5.4);
    \draw[return] (12.5,-6) -- node[above, font=\tiny\bfseries] {8: route\_id} (5,-6);
    
    \draw[message] (5,-6.8) -- node[above, font=\tiny\bfseries] {9: sha256(data)} (5,-7.1);
    \draw[message] (5,-7.5) -- node[above, font=\tiny\bfseries] {10: createRoute(hash)} (15,-7.5);
    \draw[return] (15,-8.3) -- node[above, font=\tiny\bfseries] {11: tx\_hash, block} (5,-8.3);
    
    \draw[message] (5,-9) -- node[above, font=\tiny\bfseries] {12: update\_one(tx)} (12.5,-9);
    
    \draw[return] (5,-9.8) -- node[above, font=\tiny\bfseries] {13: route\_response} (2.5,-9.8);
    \draw[return] (2.5,-10.3) -- node[above, font=\tiny\bfseries] {14: display map} (0,-10.3);
    
\end{tikzpicture}
}
\caption{Diagramme de séquence - Processus complet d'optimisation VRP avec enregistrement blockchain}
\label{fig:seq_vrp}
\end{figure}

\subsection{Diagramme de Séquence : Vérification de l'Intégrité Blockchain}

\begin{figure}[H]
\centering
\begin{tikzpicture}[
    actor/.style={rectangle, draw=primaryblue, fill=primaryblue!15, minimum width=2cm, minimum height=0.8cm, font=\footnotesize\bfseries, rounded corners=3pt},
    lifeline/.style={dashed, thick, gray!50},
    message/.style={-{Stealth[length=2.5mm]}, thick, primaryblue},
    return/.style={-{Stealth[length=2.5mm]}, dashed, thick, successgreen},
    activation/.style={fill=successgreen!20, draw=successgreen},
    alt/.style={rectangle, draw=gray, thick, rounded corners=3pt},
    altlabel/.style={font=\tiny\bfseries, fill=gray!20, inner sep=2pt}
]
    % Actors/Objects
    \node[actor] (user) at (0,0) {Utilisateur};
    \node[actor] (front) at (3.5,0) {Frontend};
    \node[actor, fill=successgreen!15, draw=successgreen] (back) at (7,0) {Backend};
    \node[actor, fill=warningorange!15, draw=warningorange] (db) at (10.5,0) {MongoDB};
    
    % Lifelines
    \draw[lifeline] (0,-0.4) -- (0,-9);
    \draw[lifeline] (3.5,-0.4) -- (3.5,-9);
    \draw[lifeline] (7,-0.4) -- (7,-9);
    \draw[lifeline] (10.5,-0.4) -- (10.5,-9);
    
    % Activation box
    \draw[activation] (6.85,-2) rectangle (7.15,-6.5);
    
    % Messages
    \draw[message] (0,-1) -- node[above, font=\tiny\bfseries] {1: click "Vérifier"} (3.5,-1);
    \draw[message] (3.5,-2) -- node[above, font=\tiny\bfseries] {2: GET /routes/\{id\}/verify} (7,-2);
    \draw[message] (7,-2.8) -- node[above, font=\tiny\bfseries] {3: find\_one(route\_id)} (10.5,-2.8);
    \draw[return] (10.5,-3.5) -- node[above, font=\tiny\bfseries] {4: route\_data} (7,-3.5);
    
    % Processing note
    \node[rectangle, draw=secondarypurple, fill=secondarypurple!10, font=\tiny, text width=3cm, rounded corners=2pt, align=left] at (10,-5) {
        stored = route.data\_hash\\
        current = sha256(data)\\
        result = (stored == current)
    };
    
    % Alt fragment
    \draw[alt, fill=blue!3] (2.8,-6.8) rectangle (11,-8.8);
    \node[altlabel] at (3.3,-6.95) {alt};
    \draw[dashed, gray] (2.8,-7.7) -- (11,-7.7);
    
    \node[font=\tiny, anchor=west] at (3,-7.3) {[hashes égaux]};
    \draw[return] (7,-7.4) -- node[above, font=\tiny\bfseries, fill=successgreen!20] {\{verified: true, status: "VALIDE"\}} (3.5,-7.4);
    
    \node[font=\tiny, anchor=west] at (3,-8) {[hashes différents]};
    \draw[return, warningorange] (7,-8.4) -- node[above, font=\tiny\bfseries, fill=warningorange!20] {\{verified: false, status: "MODIFIÉ"\}} (3.5,-8.4);
    
    % Final return
    \draw[return] (3.5,-8.8) -- node[above, font=\tiny\bfseries] {5: afficher résultat} (0,-8.8);
    
\end{tikzpicture}
\caption{Diagramme de séquence - Vérification de l'intégrité via blockchain}
\label{fig:seq_verify}
\end{figure}

%---------------------------------------------------------------------
\section{Conception du Schéma de Base de Données}
%---------------------------------------------------------------------

\subsection{Collections MongoDB}

RouteChain utilise MongoDB, une base de données NoSQL orientée documents. Les principales collections sont décrites ci-dessous.

\begin{table}[H]
\centering
\caption{Structure de la collection \texttt{drivers}}
\begin{tabular}{|l|l|l|}
\hline
\textbf{Champ} & \textbf{Type} & \textbf{Description} \\
\hline
\_id & ObjectId & Identifiant MongoDB \\
driver\_id & String & Identifiant unique (DRV\_YYYYMMDD\_HHMMSS) \\
email & String & Adresse email (unique) \\
password\_hash & String & Mot de passe hashé (bcrypt) \\
first\_name & String & Prénom \\
last\_name & String & Nom de famille \\
role & String & Rôle (admin ou driver) \\
created\_at & DateTime & Date de création \\
\hline
\end{tabular}
\end{table}

\begin{table}[H]
\centering
\caption{Structure de la collection \texttt{routes}}
\begin{tabular}{|l|l|l|}
\hline
\textbf{Champ} & \textbf{Type} & \textbf{Description} \\
\hline
\_id & ObjectId & Identifiant MongoDB \\
route\_id & String & Identifiant unique (ROUTE\_YYYYMMDD\_HHMMSS) \\
route\_name & String & Nom de la tournée \\
driver\_id & String & Référence au chauffeur \\
status & String & Statut (optimized, in\_progress, completed) \\
depot\_location & Object & Coordonnées du dépôt \\
delivery\_points & Array & Liste des points de livraison \\
optimization\_result & Object & Résultats de l'optimisation \\
blockchain\_tx\_hash & String & Hash de la transaction blockchain \\
blockchain\_block & Integer & Numéro du bloc \\
data\_hash & String & Hash SHA-256 des données \\
created\_at & DateTime & Date de création \\
completed\_at & DateTime & Date de complétion \\
\hline
\end{tabular}
\end{table}

%---------------------------------------------------------------------
\section{Architecture du Smart Contract}
%---------------------------------------------------------------------

\subsection{Structure du Contrat RouteRegistry}

Le smart contract \texttt{RouteRegistry} est déployé sur la blockchain Ethereum (Ganache en développement) et assure l'immuabilité des enregistrements de tournées.

\begin{figure}[H]
\centering
\begin{tikzpicture}[
    contract/.style={rectangle, draw=secondarypurple, fill=secondarypurple!10, thick, minimum width=10cm, rounded corners=8pt},
    section/.style={rectangle, draw=gray, fill=white, thick, minimum width=9cm, rounded corners=5pt, font=\footnotesize, align=left},
    title/.style={font=\large\bfseries, secondarypurple}
]
    % Contract box
    \node[contract, minimum height=8cm] (c) at (0,0) {};
    \node[title] at (0,3.5) {<<contract>> RouteRegistry};
    
    % State variables
    \node[section, anchor=north] at (0,3) {
        \textbf{State Variables}\\[2pt]
        \texttt{mapping(string => RouteRecord) routes}\\
        \texttt{string[] routeIds}
    };
    
    % Functions
    \node[section, anchor=north] at (0,1.3) {
        \textbf{Functions}\\[2pt]
        \texttt{+ createRoute(routeId, dataHash, status, distance, count)}\\
        \texttt{+ updateRoute(routeId, newHash, newStatus, completedCount)}\\
        \texttt{+ getRoute(routeId) : RouteRecord}\\
        \texttt{+ verifyRoute(routeId, dataHash) : bool}\\
        \texttt{+ getRouteCount() : uint256}
    };
    
    % Events
    \node[section, anchor=north, fill=successgreen!10, draw=successgreen] at (0,-1.2) {
        \textbf{Events}\\[2pt]
        \texttt{event RouteCreated(routeId, dataHash, timestamp, driver)}\\
        \texttt{event RouteUpdated(routeId, newHash, newStatus, timestamp)}\\
        \texttt{event RouteCompleted(routeId, finalHash, completedCount)}
    };
    
\end{tikzpicture}
\caption{Structure du smart contract RouteRegistry}
\label{fig:smart_contract}
\end{figure}

%---------------------------------------------------------------------
\section{Conclusion}
%---------------------------------------------------------------------

Ce chapitre a présenté la conception complète du système RouteChain à travers différents niveaux d'abstraction. L'architecture trois-tiers garantit une séparation claire des responsabilités, tandis que les diagrammes UML offrent une vision globale et détaillée du fonctionnement du système.

Les diagrammes de séquence mettent particulièrement en évidence les deux aspects innovants de l'application : le processus d'optimisation VRP avec ses multiples composants et l'interaction avec la blockchain pour la traçabilité des données.

Le chapitre suivant sera consacré à la réalisation technique, présentant les technologies utilisées et les interfaces développées.
