%=====================================================================
%                    INTRODUCTION GÉNÉRALE
%=====================================================================

\chapter*{Introduction Générale}
\addcontentsline{toc}{chapter}{Introduction Générale}
\markboth{Introduction Générale}{Introduction Générale}

\vspace{1cm}

Dans un contexte économique mondial marqué par l'essor fulgurant du commerce électronique et l'augmentation constante des attentes des consommateurs en matière de rapidité et de fiabilité des livraisons, l'optimisation logistique est devenue un enjeu stratégique majeur pour les entreprises. La gestion efficace des tournées de véhicules, communément désignée sous l'acronyme VRP (\textit{Vehicle Routing Problem}), représente l'un des défis les plus complexes et les plus étudiés dans le domaine de la recherche opérationnelle.

Parallèlement à ces préoccupations d'efficacité opérationnelle, une nouvelle exigence s'impose progressivement dans le secteur de la logistique : la traçabilité et la transparence des opérations. Les clients, qu'ils soient particuliers ou professionnels, souhaitent désormais disposer d'une visibilité complète sur le parcours de leurs colis, depuis l'entrepôt jusqu'à leur porte. Cette demande de transparence s'accompagne d'un besoin croissant de garanties quant à l'intégrité des données de livraison, notamment dans un contexte où les litiges liés aux preuves de livraison sont fréquents.

C'est dans ce double contexte d'optimisation et de transparence que s'inscrit le projet \textbf{RouteChain}. Cette application web full-stack propose une solution innovante combinant deux technologies de pointe : les algorithmes d'optimisation de tournées de véhicules, implémentés via la bibliothèque Google OR-Tools, et la technologie Blockchain, permettant d'assurer l'immuabilité et la vérifiabilité des enregistrements de livraison.

L'objectif principal de RouteChain est de fournir aux entreprises de livraison un outil complet leur permettant non seulement d'optimiser leurs itinéraires de manière à minimiser les distances parcourues et les temps de trajet, mais également de constituer une preuve irréfutable de chaque livraison effectuée grâce à l'enregistrement des données sur une blockchain Ethereum.

Le présent rapport est structuré en cinq chapitres principaux. Le \textbf{premier chapitre} présente le cadrage du projet, incluant le contexte, l'étude des solutions existantes et la proposition de notre solution. Le \textbf{deuxième chapitre} détaille la spécification des besoins fonctionnels et non fonctionnels. Le \textbf{troisième chapitre} expose la conception du système à travers différents diagrammes UML. Le \textbf{quatrième chapitre} présente la réalisation technique et les technologies utilisées. Enfin, le \textbf{cinquième chapitre} propose une discussion sur les défis rencontrés et l'évaluation du système.
